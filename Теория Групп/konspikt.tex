\documentclass[11pt,a4paper]{book}
\usepackage[utf8]{inputenc}
\usepackage[russian]{babel}
\usepackage[OT1]{fontenc}
\usepackage{amsmath}
\usepackage{amsfonts}
\usepackage{amssymb, amsthm}
\usepackage{graphicx}
\usepackage{mathrsfs, dsfont} % специальные шрифты, по типу \mathscr или \dsfont
\usepackage{comment} %для многострочных комментариев
\usepackage{xcolor} %для гиперссылок в тексте и их цвета
\usepackage{hyperref}
\author{Коченюк Анатолий}
\title{Конспект по теории групп}
\newtheorem{Def}{Определение}[chapter]
\newtheorem{Th}{Теорема}[chapter]
\newtheorem{Lm}{Лемма}[chapter]
\newtheorem{Pm}{Задача}[chapter]
\begin{document}
\maketitle
Борис Лаевский -- преподователь

\section{Перестановка}
конечное множество $N = \{1, 2, ..., n\}$

\begin{Def}
Перестановкой будем называть биективную функцию 
$
\alpha: N \to N
$
$$
\alpha = 
\begin{pmatrix}
1 & 2 &\hdots& n\\
\alpha(1) & \alpha(2)& \hdots & \alpha(n)
\
\end{pmatrix}
$$
\end{Def}

\begin{Def}
Тождественной перстановкой будем обозначать перестановку, оставляющюю все элементы на своих местах
\end{Def} 

Рассмотрим композицию двух перестановок 

$$
a = 
\begin{pmatrix}
1 & 2 & 3 & 4\\
3 & 2 & 1 & 4
\end{pmatrix}
$$
$$
b = 
\begin{pmatrix}
1 & 2 & 3 & 4\\
1 & 3 & 2 & 4
\end{pmatrix}
$$

$$
ab = a(b(X))
\begin{pmatrix}
1 & 2 & 3 & 4\\
1 & 3 & 2 & 4\\
3 & 1 & 2 & 4\\
\end{pmatrix}
$$

\begin{Def}
$S_n $ --- множество всех перестановок множества из n элементов
\end{Def}
множество перестановок --- группа Если она обладает следущими свойствами:
\begin{enumerate}
	\item $\forall a, b, c \in S_n \quad (ab)c = a(bc)$
	\item $\forall a \in S_n \quad ea = ae = a$
	\item $\forall a \in S_n \quad \exists a^{-1} \in S_n: aa^{-1} = a^{-1}a = e$
\end{enumerate}

\begin{Def}
$$
a =
\begin{pmatrix}
1 & 2 & 3 & \hdots & n\\
\alpha(1) & \alpha(2) & \alpha(3) & \hdots & \alpha(n)
\end{pmatrix}
$$

Будем называть перстановку а k-циклом, если:
\begin{enumerate}
	\item $a(i) = i+1 \quad i = \overline{1, k}$
	\item $a(k) = 1$ 
\end{enumerate}
\end{Def}

Пример:
$$
a =
\begin{pmatrix}
1 & 2 & 3 & 4\\
2 & 3 & 4 & 1
\end{pmatrix}
$$

$$
a = (2, 3, 4, 1)
$$

\begin{Def}
Транспозиция --- цикл длины 2
\end{Def}

\begin{Pm}
Сколко различных 57-циклов в $S_{57}$?

Ответ : $56!$

Решение
(n) -- количество

$$
\begin{pmatrix}
1& 2 & 3 & \hdots & 56 & 57\\
(56) & (55) & (54)& \hdots & (1) & (1) 
\end{pmatrix}
$$

Для первого нельзя поставить 1 $\Rightarrow$ осталось 56
Для второго нельзя поставить прошлый элемент и 2-ой
\end{Pm}

\begin{Pm}
Доказать, что любые два независимых цикла коммутируют (коммутативны относительно умножения) 
\end{Pm}
\begin{proof}

\end{proof}
ДЗ:
\begin{itemize}
	\item $S_5 $ количество циклов и транспозиций
	\item При каких условиях произведение двух транспозиций является циклом 
	\item[*] При каких условиях произведение двух циклов является циклом
\end{itemize}

\begin{enumerate}
	\item $S_5$
	
	Транспозиций -- ((5), (4)) -- 5*4/2 = 10
	
	3-циклов -- $C_5^3*2 = 24$
	
	4-циклов -- $C_5^4*3! = 5$ 
	
	5-циклов -- $C_5^5*4! = 24$
	
	Всего -- 10+24+5+4 ()
	
	65 -- мало
	\item a, b -- транспозиции ab -- цикл, если  они зависимы, т.е. изменяют один и тот же элемент в множестве.
	\item a, b -- циклы ab -- цикл, когда они зависимы ---||---
\end{enumerate}

$C_n^k = \dfrac{n!}{k!(n-k)!}$ количество выбрать из n элементов группы по k элементов, не обращая внимания на порядок

\begin{Th}
	Любая перестановка представима в виде произведения (n-1) транспозиции
\end{Th}

\begin{proof}
	д/з
\end{proof}

$S_1 \quad 
\begin{pmatrix}
	i_1 & i_2 & i_3 & \hdots & i_n\\
	j_1 & j_2 & j_3 & \hdots & j_n\\
\end{pmatrix}
$. (k, l) -- беспорядок, если $i_k< i_l$ и $j_k> j_l$

Перестановка чётная, если в каноническом виде (числа сверху расположены по порядку) количесвто беспорядков чётное.

Также перестановка чётная, если в не каноническом виде сумма количеств беспорядков сверху и снизу чётное
\begin{Th}
Докажем эквивалентность этих определений
\end{Th}

\begin{proof}
.\\
	Для доказательства введём ещё одно определение, которое эквивалентно и тому и другому.
	
	перестановка $b$ -- чётная, если количество пар чисел $(k, l)$ таких, что $(i_k>i_l)\&(j_k<j_l)$ 
	
	1 и 3 эквивалентны:

	применим 3 к каноническому виду. тогда мы получим, что считаем то же самое
	
	2 и 3 эквивалентны:
	
	фиксируем $k, l \in \mathds{N}$  не умаляя общности, скажем, что $k<l$
	
	\begin{enumerate}
		\item $(i_k<i_l) \& (j_k>j_l)$ тогда эта пара даёт +1 к количеству беспорядков по 3-му определению. А по 2-му она не добавляет ни одной к верхней строчке и одну к нижней. Таким образом тоже +1
		
		Случай с обратными знаками аналогичен, просто беспорядок добавляется к верхней строчке, а не к нижней
		\item $(i_k<i_l) \& (j_k<j_l)$аналогично
	\end{enumerate}	 
\end{proof}

\begin{itemize}
	\item Доказать, что при умножении на транспозицию справа или слева чётность меняется
	\begin{proof}
		$a =
	\begin{pmatrix}
		1 & 2 & \hdots & i & \hdots & j & \hdots &n\\
		a_1 & a _2 & & a_i & & a_j & & a_n
	\end{pmatrix}$
		
	$t = (i, j)$
	
	$at =
	\begin{pmatrix}
		1 & 2 & \hdots & i & \hdots & j & \hdots &n\\
		a_1 & a _2 & & a_j & & a_i & & a_n
	\end{pmatrix}$
	
	Если между $i$ и $j$  $k$ перестановок, то количество беспорядков, возникающих при переставлении $a_i$ и $a_j$ = 2$k$-1. это число нечётное $\Rightarrow$ чётность количесва беспорядков меняется.
	\end{proof}
	\item имеется 2 перестановки a, b. как определяется чётность через чётность a и чётность b
	\begin{proof}
		$a = k$ транспозиций
		
		$b = l$ транспозиций
		
		$ab = k+l$ транспозиций
	\end{proof}
\end{itemize}

в $S_n$ равное количество чётных и нечётных
ДЗ:
\begin{Th}
	$(ab)^{-1}= b^{-1}a^{-1},~a, b \in S_n$
	\begin{Def}[возведение перстановок в степень]
		$a \in S_n$
		\begin{enumerate}
			\item $a^0 = e = id$
			\item $a^{k+1} = a^ka$
		\end{enumerate}
	\end{Def}
	
	Доказать, что $a^{-k} = (a^{-1})^{k} \quad \equiv a^{-k}$ -- обратное к $a^k$
\end{Th}

\begin{proof}
	$a^{-k}$ -- обратное к $k$. Нужно доказать, что $a^{-k}=(a^{-1})^k$
	
	$a^k\cdot(a^{-1})^k=e$?
	
	$$a^k \cdot (a^{-1})^k= a \cdot \hdots\cdot a\cdot a\cdot a^{-1}\cdot a^{-1}\cdot\hdots\cdot a=a^k\cdot(a^{-1})^k=$$
	$$= a\cdot\hdots\cdot a\cdot e\cdot a^{-1}\cdot\hdots\cdot a^{-1}=\hdots a\cdot a\cdot a^{-1}a^{-1}=a\cdot e\cdot a^{-1}=a\cdot a^{-1}=e$$
\end{proof}


Найти все $a \in S_n$, что $\forall b \in S_n$
\begin{enumerate}
	\item $ba=b$
	
	$b^{-1}ba = b^{-1}b$
	
	$a=e=id$
	\item $ba = ab$
	\item $ba = ab^{-1}$
\end{enumerate}



Свойства стееней:
\begin{enumerate}
	\item $a^{k+l}=a^k*a^l$
	\item $a^{kl}=(a^k)^l$
	\item Если $ab = ba$, то $ab^k = a^kb^k$ очевидно
\end{enumerate}

$\forall (k, m) \& \forall a \quad a(k, m)=(k, m)a\Rightarrow a = e$\

Мы знаем, что любая перстановка представима в виде произведения конечного числа циклов. И, возведя каждый из них в степень равную их длинне получим $id=e$. Мы можем возвести всю перстановку в степень равную НОК длин её составляющих, таким образом каждый цикл станет равен $id=e$, а их произведение будет тоже $id=e$

наименьшее такое $k$ будем называть порядком перстановки 

\begin{Th}
	$S_n \quad a \in S_n \quad a = c_1\hdots c_i$
	
	Тогда порядок $a$равен НОК$|c_1|\dots|c_n|$, где $|c|$-- порядок $c$ 
\end{Th}

\begin{proof}
	$k = $ НОК$(|c_1|\hdots|c_i|)$
	
	$k\vdots |c_j|\quad \forall j = \overline{1, i}$
	
	$c_j^k = e$
	
	$a = c_1\hdots c_i$
	
	$a^k = (c_1\hdots c_i)^k=c_1^k\hdots c_i^k=e\hdots e = e$
	
	Попробуем взять $l<k$такое, что 
	$l\not\vdots |c_j|\quad c_j^l\neq e\quad a^e \neq e$
\end{proof}

Пусть $k$ -- порядок $a$, тогда $a^n=e\Longleftrightarrow n \vdots k$

\section{Группы}
$\cdot : MxM \to M$ -- бинарная операция

$(G, \cdot)$ -- группа если:
\begin{enumerate}
	\item $a(bc) = (ab)c$
	\item $\exists c \in G: \forall a \in G \quad ac = ca = a$
	\item $\forall a \in G \quad \exists a^{-1}\in G: aa^{-1}= a^{-1}a = e$
\end{enumerate}

$(S_n, \cdot)$ -- группа

Количество элементов в $G = |G| = $ порядок $G$
\begin{enumerate}
	\item $(\mathds{Z}, +)$ да
	\item $(\mathds{Z}, -)$ нет
	\item $(\mathds{N}, *)$ нет
	\item Множество чётных чисел относительно сложения 
	\item Множество нечётных чисел относительно сложения
	\item $(f:X\to X, \cdot)$
	\item $(B(A), \cup)$
	\item $(B(A), \cap)$
	\item $(B(A), \backslash)$
	\item $(B(A), \triangle)$
\end{enumerate}

$f: G\to H$ -- изоморфизм, если :
\begin{enumerate}
	\item $f$ -- биекция
	\item $\forall x, y \in G\quad f(x\cdot y) = f(x)\cdot g(x)$
\end{enumerate}

 и трёх ээлементов
Дз
попарные неизоморфные группы из одного и двух элементов (3 и 4-х тоже)
\end{document}