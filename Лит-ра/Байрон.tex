\documentclass[10pt,a4paper]{article}
\usepackage[utf8]{inputenc}
\usepackage[russian]{babel}
\usepackage[OT1]{fontenc}
\usepackage{amsmath}
\usepackage{amsfonts}
\usepackage{amssymb}
\usepackage{graphicx}
\author{Коченюк Анатолий}
\title{Биография Байрона}
\begin{document}
\maketitle
\begin{center}
\textbf{Джордж Ноэл Гордон Байрон}
\end{center}

1788 -- 1824

Прочитал богатую библиотеку\\

1805 -- начало обучения в Тринити колледж. Не очень интересовался наукой и предпочитал ей плавание (однажды проплыл пролив в средиземном море). Учился плохо. Прославился рыцарскими качествами. 

1806 -- пишет стихи "Стихи на слуай"

1807 -- "часы досуга"

1809 -- уезжает из Англии

1811 -- возвращается из путешествия (посетив Испанию, Албанию, Турцию и Малую Грузию)

1812 -- первые две песни: "Паломничество Чайльд Гарольда"

1817 -- третья песня

1818 -- четертая песня, Швейцария и Италия

Воевал за Испанию

Стал национальным героем Греции

Байронизм -- романтическое течение литературы, связанное с главным Байроническимс героем ()

Отдал свои мемуары Томасу М.. Его мемуары оказались слишком честными и их сожгли.

Был пансексуалом (влечёт любой пол)
\end{document}
