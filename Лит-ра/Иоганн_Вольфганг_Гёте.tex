\documentclass[10pt,a4paper]{article}
\usepackage[utf8]{inputenc}
\usepackage[russian]{babel}
\usepackage[OT1]{fontenc}
\usepackage{amsmath}
\usepackage{amsfonts}
\usepackage{amssymb}
\usepackage{graphicx}
\author{Коченюк Анатолий}
\title{Иоганн Вольфганг Гёте}
\begin{document}
\maketitle

Родился 29.08.1749

1756 -- 1758. Франкфурд-на-Майне
Позже перешёл на дрмашнее обучение. Обучался с сестрой Корнелией. В детстве выучил множество различных языком, в том числе древних

1765 -- Лейпцигский (Ляйпцигский) университет ул. Нойкмарт

Был свидетелем семилетней войны, отделения Америки от Англии, военных действий Наполеона и ему было роще об этом говорить, чем тем, кто узнавал об этом из книг

Влюбляется в Кэтхен Шёнкпоф, дочь трактирщика. пишет весёлые стихи

1768 -- возвратился домой, не получив научную степень. Во время болезни увлекался Религией. Появляется первая комедия -- "Соучастники"

1770 -- возвращается в Страсбургский университете. Знакомится с Ленцем, Шиллерром.

1771 -- представил свою диссертацию, которая вызвала множество противоречий у Церкви

1775 -- помолвкка с дочерью Банкира Лили Шёнеман. Переезжаент в Веймар, где живёт до конца жизни.

1780е -- путешествует по Италии

1813 -- 1817 Новый Роман с Кристиной ... От неё у Гёте были дети

От сногих интеллектуалов отличался тем, что не был охвачен патриотизмом.

1808 -- виделся с Наполеоном. Наблюдал за ним, как за интересным явлением.
\end{document}