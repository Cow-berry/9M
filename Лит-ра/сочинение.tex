  \documentclass[10pt,a4paper]{article}
  \usepackage[utf8]{inputenc}
  \usepackage[russian]{babel}
  \usepackage[OT1]{fontenc}
  \usepackage{amsmath}
  \usepackage{amsfonts}
  \usepackage{amssymb}
  \usepackage{graphicx}
  \author{Коченюк Анатолий}
  \title{Истинность искусства}
  \begin{document}
  	\maketitle

  
  Я считаю, что у всех или хотя бы у многих есть изначальная предрасположенность к какому-либо типу деятельности, которая называется талантом. Но его можно вовремя не заметить и он пропадёт.
   

  
  Рассмотрим, что произошло с талантом главного героя [Чарткова]  в первой части повести Гоголя "Портрет".
  
  В начале повести описывается человек, у которого есть талант художника. Он творит для себя, а не для других. Творит так, как подскажет ему талант. Живёт он бедно, еле-еле оплачивает себе жильё, голодает.
  
  Но затем внезапно Чартков обнаруживает клад, спрятанный в купленной им на последние деньги картине. И это меняет его жизнь. Художник покупает квартиру, покупает фрак, идёт в театр, в кондитерскую, словом покупает огромное количество того, без чего мог бы спокойно обойтись за один день. Начинает планировать как он потратит эти деньги в будущей жизни. И после всего этого возвращается к жизни художника. Но не того, который пишет то, что ему захочется, а того, который пишет портреты на заказ, потому что таких заказов  больше всего.
  
  В процессе написание портретов Чарткова постоянно поправляют и советуют. Из-за этого он страдает как художник, которому не дают писать так, как он хочет. В итоге Чартков привыкает к постоянным советам и постепенно перестаёт проявлять свой талант. Позже он и вовсе только начинал, а заканчивать отдавал своим ученикам.
  
  Так он и потерял талант.
  
  И только когда Чартков уже в преклонном возрасте получает письмо от академии художеств и отправляется посмотреть на картину, присланную из Италии. Как только он её видит, зависть охватывает его. Чартков видит высшее проявление таланта, который он растерял в однообразии желаний народа. 
  
  В качестве вывода хочу сказать, что истинное искусство это результат деятельности человека, предрасположенного к этому, т.е. с талантом в определённой сфере деятельности.
  
  
  	
  \end{document}