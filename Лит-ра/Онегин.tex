\documentclass[12pt,a4paper]{article}
\usepackage[utf8]{inputenc}
\usepackage[russian]{babel}
\usepackage[OT1]{fontenc}
\usepackage{amsmath}
\usepackage{amsfonts}
\usepackage{amssymb}
\usepackage{graphicx}
\usepackage[left=2cm,right=2cm,top=2cm,bottom=2cm]{geometry}
\author{Коченюк Анатолий}
\title{Евгений Онегин}
\begin{document}
\maketitle

роман -- описание частной жизни

роман в \textbf{стихах}. Он решил поэксперементировать.

+разделено на "Онегинские строфы" все они взаимосвзаны и в то же время мы можем читать их как отдельные тексты/стихотворения

Онегин -- Пушкин в юности. похож на Байрона

когда Онегин переехал в деревню, Пушкин сочувствует ему

Пушкин видит много противоречий, но не хочет их исправлять, т.к. он бы получился плоский нереальный

Россия всё ещё деревня

2 глава -- синтементализм

английский романтик -- герой, который себя ведёт высокомерно апо отношению к другим без типичного двоемирия (перемещается географически)

немецкий романтик -- есть двоемирие

Пушкин и Ленский оба приехали в деревню


Ленский -- романтик из Германии. Либерал. 

Германия туманная всмысле отсылки к чему-то романтическому. 

Татьяна видит человека, в котрого её следует влюбиться -- Ленского. и влюбляется

3 глава XXIV строфа -- россыпь характеристик Татьяны. Доверчивость, открытость, ум, живая воля

воображение мятежное -- внутренняя свобода

готовиться к сочнению по 1-2 главам

читать 4-ю



\end{document}