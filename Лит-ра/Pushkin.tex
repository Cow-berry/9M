\documentclass[12pt,a4paper]{article}
\usepackage[utf8]{inputenc}
\usepackage[russian]{babel}
\usepackage[OT1]{fontenc}
\usepackage{amsmath}
\usepackage{amsfonts}
\usepackage{amssymb}
\usepackage{graphicx}
\author{Коченюк Анатолий}
\title{Пушкин}
\begin{document}
\maketitle

\textbf{Первый период} 1813 -- 1817
Царскосельсикй лицей. ''Воспоминания о Царском лицее''

Прозвище -- Француз. 

Словарный запас Русского языка -- 26 000 слов.

Учился хорошо только по русской словесности.

Работа не приносила никакого удовольствия.




\textbf{Второй период} 1817 -- 1820
Становится популярным. Ода ''Вольность''

Один из первых стал популярен не только в России, но и за границей.

"Руслан и Людмила"

"К Чаадаеву"



\textbf{Третий период} (1820 - 1824)

www.Южная\_ ссылка.ru -- южные поэмы (''Кавказский пленник'', первая глава ''Евений Онегин'')

1822 - ''Кавказский пленние''

\textbf{Четвёртый период} (1824 -- 1826)

Ссылка в Михайловске. Были задуманы некоторые произведения

''Повести Белкого''

''Борис Годунов''

\textbf{Пятый период} (1826 -- 1830)

реализм

\textbf{Болдинская осень} 1830 -- 1837

"Медный всадник"

"Капитанская дочка"




//////////////////////////////////////////////

1824 -- переход к реализм
\end{document}