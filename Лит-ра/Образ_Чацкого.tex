\documentclass[12pt,a4paper]{article}
\usepackage[utf8]{inputenc}
\usepackage[russian]{babel}
\usepackage[OT1]{fontenc}
\usepackage{amsmath}
\usepackage{amsfonts}
\usepackage{amssymb}
\usepackage{graphicx}
\author{Коченюк Анатолий}
\title{Образ Чацкого}
\begin{document}
\maketitle

Рассмотрим образ Чацкого  из произведения "Горе от ума". 
Чацкий -- герой романтик. Он много путешествовал и хочет изменить мир (в связи с чем всем что-нибудь советует)

''
Хлёстова\\
Туда же из смешливых;\\
Сказала что-то я: он начал хохотать.\\
Молчалин\\
Мне отсоветовал в Москве служить в Архивах.\\
Графиня внучка\\
Меня модисткою изволил величать!\\
Наталья Дмитриевна\\
А мужу моему совет дал жить в деревне.\\
''


Чацкий любит Софию. До его путешествий это было взаимно, но после его возвращения она стала больше любить Молчанина. Чацкий начинает ревновать 

''Что я Молчалина глупее? Где он, кстати?\\
Еще ли не сломил безмолвия печати?\\
Бывало песенок где новеньких тетрадь\\
Увидит, пристает: пожалуйте списать.\\
А впрочем, он дойдет до степеней известных,\\
Ведь нынче любят бессловесных.''
\\
А в конце оказывается, что Молчанин лишь играл возлюбленного Софии, хотя на самом деле являлся совершкнно другим человеком.
''
Молчалин\\
Мне завещал отец:\\
Во-первых угождать всем людям без изъятья;\\
Хозяину, где доведется жить,\\
Начальнику, с кем буду я служить,\\
Слуге его, который чистит платья,\\
Швейцару, дворнику, для избежанья зла,\\
Собаке дворника, чтоб ласкова была.72\\
Лиза\\
Сказать, сударь, у вас огромная опека!\\
Молчалин\\
И вот любовника я принимаю вид\\
В угодность дочери такого человека…\\
''
\\

Также Фамусов был ослеплён своим умом.	\\
Он хотел просветить всех (см. первую цитату)\\
И удивлялся почему его все не любят (см. вторую цитату)\\

Ещё его все начали считать сумасшедшим. 

Сначала один из гостей называет его сумасшедшим в связи с странными советами от него.

''
Натал. Юрьевн.\\
А мужу моему совет дал жить в деревне.\\
Загарецк.\\
Безумный по всему.\\
Графн. внучка\\
Я видела из глаз.\\
Фамусов\\
По матери пошел, по Анне Алексевне,\\
Покойница с ума сходила осемь раз.\\
Хлёстова\\
На свете дивные бывают приключенья!\\
В его лета с ума спрыгнул!\\
Чай пил не по летам.\\
Княгиня\\
О! верно.\\
Графиня внучка\\
Без сомненья.\\
Хлёстова\\
Шампанское стаканами тянул\\
Натал. Юрьевн.\\
Бутылками-с, мы замечали сами.\\
Загарецк.(с жаром)\\
Наталья Юрьевна.190 Ведрами, да-с ведрами.\\
Фамусов\\
Ну вот! великая беда,\\
Что выпьет лишнее мущина!\\
Ученье, — вот чума, ученость, — вот причина,\\
Что ныньче пуще, чем когда\\
Безумных развелось людей, и дел, и мнений.\\
Хлёстова\\
И впрямь с ума сойдешь от этих от одних\\
От пансионов, школ, лицеев, как бишь их,\\
Да от ланкарточных взаимных обучений.\\
Княгиня\\
Нет, в Петербурге институт\\
Пе-да-го-гический,191 так кажется зовут:\\
Там упражняются в расколах и в безверьи\\
Профессоры!! у них учился наш родня,\\
И вышел! хоть сей час в аптеку в подмастерьи,\\
От женщин бегает, и даже от меня!\\
Чинов не хочет знать! Он химик, он ботаник,\\
Князь Федор мой племянник.\\
''

Эту идею сразу же подхватывают остальные, прилагая свои аргументы. Люди, чтобы убедится в новом слухе спрашивают остальных о нём, тем самым его распространяя. В конце пьесы о сумасшествии Чацкого знают все и глубоко в этом убеждены, а те, кто нет, подвержены давлению

''
Загарецкий\\
Княжны, пожалуйте, скажите ваше мненье,\\
Безумный Чацкий или нет?\\
1-я кнжна\\
Какое ж в этом есть сомненье?\\
2-я кжна\\
Про это знает целый свет.\\
3-я кжна\\
Дрянские, Хворовы, Варлянские, Скачковы…\\
4-я кжна\\
Ах! вести старые, кому оне новы?\\
5-я кжна\\
Кто сомневается?\\
Загарецк.\\
Да вот не верит.\\
6-я кжна\\
Вы!\\
Все вместе\\
Мсьё Репетилов! вы! Мсьё Репетилов что вы!\\
Да как вы! можно ль против всех!\\
Да почему вы! стыд и смех.\\
''



Подводя итоги, хочу сказать, что по моему мнению Фамусов хороший человек, хоть и несколько гордый за свой ум, ведь он хотел своими советами помочь людям. Также мне жалко этого героя, ведь его любимая полюбила другого, а его самого назвали сумасшедшим.






\end{document}