 \documentclass[10pt,a4paper]{article}
 \usepackage[utf8]{inputenc}
 \usepackage[russian]{babel}
 \usepackage[OT1]{fontenc}
 \usepackage{amsmath}
 \usepackage{amsfonts}
 \usepackage{amssymb}
 \usepackage{graphicx}
 \author{Коченюк Анатолий}
 \title{Виктор Гюго}
 \begin{document}
 \maketitle
 Родился в городе Безансон.
 Был младшим из трёх братьев (Абель и Эжен).
 Отец -- генерал Напалеоновской армии. Много с ним путешествовал.
 Посетил Марсель, Корсика, Мадрид, Италия
 
 1813 -- переезд с мамой в Париж
 1814 -- 1818 --- учится в лицее Людовика Великого
 1817 --Литературный кокурс
 1822 -- первый сборник "Оды и разные стихи"
 1830 -- 1843 -- Гюго занимается драматургией (пишет только для театра).
 1822 -- первая свадьба с Адель Фугис
 Первая жена изменила ему с его другом.
Женитьба на Жульетте (содержал Анатолий Демидов)
Пневмония\\
1815  -- стал увлекаться поэзией\\
1822 "Оды и разные стихотвороения"\\
1826 "Оды и баллады"\\
1827 -- первая драма для театра "Кромбель" НАЧЛО РОМАНТИЗМА\\
1829-- "Ориенталии	"
1829 - 1839 -- столкновение простолюдинов с буржуазной аристократией.
1851 - 1870 -- "Осенние листья", "Песни Сумерек", "Лучи и тень"
1856 2-томный сборник "Созерцания"
1852 -- "Бон-пард малый"
1878 -- "История одного преступления"\
вёл активную сциально-политическую жизнь. Прославился за пределами Франции.

1885 -- смерть

дз -- Пролог на небе до погребка Ауербаха
 \end{document}
 
 