\documentclass[12pt,a4paper]{article}
\usepackage[utf8]{inputenc}
\usepackage[russian]{babel}
\usepackage[OT1]{fontenc}
\usepackage{amsmath}
\usepackage{amsfonts}
\usepackage{amssymb}
\usepackage{graphicx}
\author{Коченюк Анатолий}
\title{Александр Грибоедов}
\begin{document}
\maketitle

Дата рождения точно не известна. Она варьируется от 1795 до 1790

Дата смерти от 1824 до 1829

Следущие даты будут от 1795 -- 1829

Был композитором, но до нас дошли только 2 его вальса.

1802 -- поступает в университетский пансион

30 января 1806 -- университет (11 лет!)

Затем увлёкся патриотическими чувсствами и поступил в армию на Корнето-Московский государственный полк 

Умирает граф Грибоедов.

20 марта 1816 -- получает оставку (потихоньку начинает писать "Горе от Ума")

Учавствоввал в двух массонских ложах (Объединённых друзей и добра)

Дуэль. В ней ему отсрелили палец

Средства матери истощились.

1823 -- отпуск, выезд в Россию

1824 -- "горе от Ума".Прошло цензуру не полностью, а только частично.

Был знаком с декабристами.

Отпуск был просрочен и ему нужно было вернутся в Тбилиси. начала поехал в Киев, потом в Крым.

июнь 1826 -- задержание в Крыму, за связь с декабристами. На допросе свё отрицал, а декабристы (его друзья) подтвердили и в итоге его выпустили.

Жил на даче у Булгарина.

Николай II выслал его лично в Персию (на Кавказ), потому что требовался энергичный дипломат.

Женится на Нино Завговадзе. Рожили вместе нескольео месяцев, но верность хранили всегда.
7 октября -- прибытие в Персию.

30 января 1829 -- нападение. Все были убиты. 

Чтоы как-то замять это дело персидское правительство отправиило гигантский алмаз


\end{document}