 \documentclass[10pt,a4paper]{article}
 \usepackage[utf8]{inputenc}
 \usepackage[russian]{babel}
 \usepackage{amsmath}
 \usepackage{amsfonts}
 \usepackage{amssymb}
 \usepackage{graphicx}
 \usepackage{esvect}
 \author{Коченюк Анатолий}
 \title{Формулы по физике}
 \begin{document}
 	\maketitle
 	\section{Равномерное движение}
 	Проекции вектора перемещения на оси координат через его начало и конец
 	$$S_x = x-x_0=\Delta x$$
 	$$S_y = y-y_0=\Delta y$$
 	Модуль вектора перемещения через его проекции на оси координат
 	$$\vv{S}=\sqrt{S_x^2+S_y^2}$$
 	Скорость при равномерном движении
 	$$\vv{V}=\dfrac{\vv{s}}{t}$$
 	Уравнение координаты тела при равномерном движении
 	$$x = x_0+V_xt$$
 	\section{Равнопеременное движение}
	Мгновенная скорость при неравномерном движении 	
 	$$\vv{V_{mn}}=\lim_{\Delta t\rightarrow 0}\dfrac{\vv{\Delta S}}{\Delta t}$$
 	Ускорение по определению
 	$$\vv{a}=\dfrac{\vv{V}-\vv{V_0}}{\Delta t}$$
 	Скорость при равнопеременном движении
 	$$\vv{V}=\vv{V_0}+\vv{a}t$$
 	Перемещение при равнопеременном движении без учёта ускорения
 	$$S_x=\dfrac{V_x+V_{0_x}}{2}t$$
 	Средняя скорость при равнопеременном движении
 	$$V_{cp}=\dfrac{S_x}{t}=\dfrac{V_x+V_{0_x}}{2}$$
 	Перемещение при равнопеременном движении буз учёта времени
 	$$S_x=\dfrac{V_x^2-V_{0_x}^2}{2a_x}$$
 	Уравнение перемещения при равноускоренном движении
 	$$x = x_0+V_{0_x}t+\dfrac{a_xt^2}{2}$$
 	$$S_x = V_{0_x}t+\dfrac{a_xt^2}{2}$$
 	\section{Относительное движение}
 	Перемещение относительно НСО (неподвижной системы отсчёта) равно векторной сумме перемещения тела относительно ПСО(подвижной..) и перемещения ПСО относительно НСО
 	$$\vv{S_{H}}=\vv{S_1}+\vv{S_2}$$
 	Такое же правило сложения работает для скоростей
 	$$\vv{V_{H}}=\vv{V_1}+\vv{V_2}$$
 	\section{Вращательное движение}
 	Вращательное движение -- равномерное движение по окружности\\
 	Угол поворота равен частному длины пройденной дуги и радиуса
 	$$\varphi = \dfrac{l}{R}$$
 	Линейная скорость равна частному длины дуги и времени её прохождения
 	$$V=\dfrac{l}{t}$$
 	Период вращения равен частному времени прохождения некоторого количества оборотов и количества этих оборотов
 	$$T=\dfrac{t}{n}$$
 	Частота вращения равна частному количества оборотов и времени, необходимого для их прохождения
 	$$\nu=\dfrac{n}{t}$$
 	Связь периода и частоты вращения
 	$$\nu=\dfrac{1}{T}$$
 	Угловая скорость
 	$$\omega=\dfrac{l}{t}\left[\dfrac{rad}{s}\right]$$
 	Связь линейной скорости и периода вращения
 	$$V=\dfrac{2\pi R}{T}$$
 	Связь линейной скорости и частоты вращения
 	$$V=2\pi R\nu$$
 	Связь между угловой скоростью и периодом вращения
 	$$\omega=\dfrac{2\pi}{T}$$
 	Связь между угловой скоростью и частотой вращения
 	$$\omega=2\pi\nu$$
 	Связь между линейной и угловой скоростями
 	$$V = \omega R$$
 	Модуль центростремительного ускорения
 	$$a_{c}=\dfrac{V^2}{R}$$
 	\section{Движение под действием силы тяжести}
 	\subsection{Свободное падение}
 	Проекция скорости на ось y
 	$$V_y=g_yt$$
 	Высота 
 	$$H_y=\dfrac{gt^2}{2}$$
	Высота без учёта времени
	$$H_y=\dfrac{V_y^2}{2g_y}$$
	Уравнение координаты
	$$y = y_0+\dfrac{gt^2}{2}$$
	Время полёта
	$$t_{no}=\sqrt{\dfrac{2h}{g}}$$
	\subsection{Вертикальное движение вверх под действием силы тяжести}
	Высота
	$$H_y = V_{0_y}t+\dfrac{g_yt^2}{2}$$
	Высота без учёта времени
	$$H_y = \dfrac{V_y^2-V_{0_y}^2}{2g_y}$$
	Уравнение координаты
	$$y = y_0 + V_{0_y}t+\dfrac{gt^2}{2}$$
	\subsection{Горизонтальное движение}
	Дальность полёта
	$$l = V_0  \sqrt{\dfrac{2h}{g}}$$
	Модуль скорости
	$$V=\sqrt{g^2t^2+V_0^2}$$
	Время полёта
	$$t_{no\lambda}= \sqrt{\dfrac{2h}{g}}$$
	Тангенс угла вхождения в поверхность
	$$\tg{\alpha}=\dfrac{gt}{V_0}$$
	$$\tg{\alpha}=\dfrac{V_y}{V_x}$$ 
	\subsection{Движение тела брошенного под углом горизонту}
	Время полёта
	$$t_{no\lambda} = \dfrac{2V_0\sin{\alpha}}{g}$$
	Дальность полёта
	$$l = \dfrac{V_0^2\sin{2\alpha}}g=\dfrac{2V_0^2\sin{\alpha}\cos{\alpha}}{g}$$
	Максимальная высота
	$$H_{max}=\dfrac{V_0^2\sin^2 {\alpha}}{2g}$$
	Время подъёма
	$$t_{no\Delta} = \dfrac{V_0\sin{\alpha}}{g}$$
	\section{Закон простарнства}
	При равнопеременном движении  можно определить расстояние, пройденное телом в n-ую секунду
	$$S_n=\dfrac{2n-1}{2}a$$
 \end{document}