 \documentclass[10pt,a4paper]{article}
 \usepackage[utf8]{inputenc}
 \usepackage[russian]{babel}
 \usepackage[OT1]{fontenc}
 \usepackage{amsmath}
 \usepackage{amsfonts}
 \usepackage{amssymb}
 \usepackage{graphicx}
 \author{7 M + 9 M}
 \title{Сценарий с диалогами}
 \begin{document}
\maketitle 
 
\textbf{Вошёл бог c подносоом полным здоровой крутой еды
}
\textbf{Вошёл Дьявол с дошираком, кинул его куда-то вбок и хотел забрать еду бога. Его заметили.
}
\newline
\newline
\textbf{Бог} -- Здравствуйте, коллега

\textbf{Дьявол} -- Как там успеваемость у вашего отличника 

\textbf{Бог} -- Всё отлично, его оценки как всегда высоки.

\textbf{Дьявол} -- Видно вы ему задания не самые сложные даёте. Ну не может же он быть таким умным.

\textbf{Бог} -- Да вы что? Иов -- лучший из лучших, умнейший из умнейших.

\textbf{Дьявол} -- А я вот не верю, что он может быть настолько хорош.

\textbf{Бог} -- Да уверяю вас, это так.

\textbf{Дьявол} -- Давайте заключим пари. Победившему безлимтный абонемент в божественную столовую. Что же касается проигравшег(о).. (пауза, на экране картинка доширака)

\textbf{Бог} -- я-то в Иове уверен, так что по рукам.
\newline
\newline
\textbf{Раздаётся звук грома. Бог и Дьявол расходятся в разнве стороны со словами (которые я забыл). Бог уходит за кулисы, а Дьявол садится близзко к ним.}
\newline
\newline
\textbf{Входит Иов. Проходит до центра в мечтательно-созерцательном состоянии (как будто сессию сдал). Дьявол один раз тонким голосом зовёт его. Иов думает, что это птички}
\newline
\newline

\textbf{Дьявол} -- Хэй, Иов!

\textbf{Иов подходит к Дьяволу в маске Бога.}

\textbf{Иов} -- Вы что-то хотели?

\textbf{Дьявол} -- Ты что, не узнаёшь меня, мой лучший ученик Иов.

\textbf{Иов} -- Простите, учитель, обознался.

\textbf{Дьявол} -- Хм… Обознался он! Так! У нас в ЛНМО несколько поменялся учебный план… Вот твои новые задания! (сует Иову в лицо пергамент с заданиями и удаляется).

\textbf{Иов} один. Читает. 

\textbf{Иов} -- Первое… (его прерывают)
\newline\newline
\textbf{Музыка. На экране фотография первого приспешника Дьявола. Подпись «МегаМатан. Осужден божественным судом и выслан из ЛНМО за чрезмерное потребление математики и использования своих знаний в гнусных целях».
Выходит. Приносит стаканчик. Говорит первую задачу. После ответа начинает исполнять номер со стаканчиками параллельно разговаривая.}
\newline\newline
\textbf{МегаМатан} -- Замыкание объединения двух множеств равно? (кидает Иову штангу)

\textbf{Иов} -- Объединению замыканий!

\textbf{МегаМатан} -- Замыкание множества Е есть пересечение?

\textbf{Иов} -- Есть пересечение всех замкнутых множеств, содержащих Е!

\textbf{МегаМатан} -- Для любого набора из трёх неотрицательных чисел верно, что?

\textbf{Иов} -- Верно то, что произведение их попарных сумм больше либо равно восьмикратному произведению набора! Ха!

(МегаМатан обескуражен, роняет свою штангу с книгами. Иов ее поднимает).

\textbf{Иов} -- А еще я знаю, что треугольник, вершинами которого служат точки пересечения медиан боковых граней тетраэдра подобен основанию тетраэдра!!! (кидает стаканчик в приспешника, тот падает под его "весом", уползает за кулисы).

Дьявол потрясает руками на авансцене.

\textbf{Дьявол} -- Что ж… Сейчас в ход пойдет тяжелая артиллерия! Тут-то Иову точно несдобровать!

\textbf{Иов} -- Второе… (сделайте все задания из Гольдфарба (тут можно принести листок, мелко исписанный текстом, похожим на дз))
\newline\newline
\textbf{После прочтения задания дьявол выводит за руку последний свой козырь. На экране фотография. Подпись 
«Undina Mortis». 
Музыка меняется. Выходит ундина. 
Начинает исполнять акробатику перед Иовом. Тот сначала растерян, но затем берёт себя в руки.}
\newline\newline
\textbf{Иов} -- Слушай, бросай-ка ты своего Дьявола и давай поступай к нам в школу! Здесь и атмосфера хорошая, и преподы отличные, да и кормят неплохо!

\textbf{Ундина} -- А и вправду! Так и сделаю! 

Иов и Ундина уходят вместе за кулисы. Дьявол на авансцене беснуется. Пытается быстро доесть то, что было на подносе. Выходит Бог.

\textbf{Бог} -- Ну что, коллега, проиграли вы пари, как я погляжу! Так что безлимитный абонемент в божественную столовую остается у меня!

\textbf{Дьявол} -- Вот чёрт.

\textbf{Бог} -- А если хочешь хорошо питаться, то устраивайся к нам в школу. 

\textbf{Дьявол} -- А что так можно?

\textbf{Бог} -- Да, конечно можно. Жду тебя завтра к 10-и.
\newline
Бог с подносом уходит. Радостный дьявол уходит в другую кулису.

КОНЕЦ
 \end{document}