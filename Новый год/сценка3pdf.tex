 \documentclass[10pt,a4paper]{article}
 \usepackage[utf8]{inputenc}
 \usepackage[russian]{babel}
 \usepackage[OT1]{fontenc}
 \usepackage{amsmath}
 \usepackage{amsfonts}
 \usepackage{amssymb}
 \usepackage{graphicx}
 \author{Коченюк Анатолий}
 \title{Сценка Иов Математики}
 \begin{document}
 \maketitle
 Сюжет частично взят из Иова.

Персонажи:
Бог — 
(Диавол) — [не выбран]
Иов — Синицын Александр (не подтверждён у режиссёра)
Рассказчик — [не выбран]

План:
Бог насылает на хорошего ученика (Иова) проклятье — зимнюю сессию.
Затем есть 2-е  проклятье — годовые экзамены.
После того как ученик проходит все испытания, он получает себе летние каникулы.

Замечания:
Начало с начала Учебного года.
Возможен спор между богом и дьяволом на щелбан.
Найти картинки, видосики, звуки.








это была задумка. Теперь я попытаюсь написать сценарий по этому

Рассказчик — Был в Лаборатории один ученик и звали его Иовом. Много похвал и пятёрок получал.

Иов вышедший на сцену показывает свои пятёрки (правда не знаю ещё как. Может листок с кучей крупно написанных на нём пятёрок).

Рассказчик — Но как-то раз над его обучением навис вопрос. Началось всё со встречи Бога с Дьяволом.

Выходят Бог с Дьяволом.

Бог — Где ты был?

Дьявол — Да вот в Лаборатории Непрерывного Математического Образования на ребят смотрел.

Бог — А видел ли ты Иова, вот хороший ученик, всегда любил меня, и никогда не хотел покинуть ЛНМО.

Дьявол — А не от того ли он хорош, что даешь ему ты пятёрок и похвал бесчисленное множество. Отбери хорошие оценки и замени на плохие, и, глядишь, по другому на тебя смотреть он будет.

Бог — Если тебе так этого хочется, то разрешаю тебе самому это устроить. Только не забирай у него знания и обучаемость.

Бог с Дьяволом уходят, и снова за сценой слышится голос рассказчика. Выходит Иов.

Рассказчик — И забрали у Иова его оценки отличные. Начали выдавать удовлетворительные и не очень.

Группа учителей выходит. Забирает у Иова пятёрки (красные) и отдаёт тройки и двойки (синие и фиолетовые).

Рассказчик — Но стойко выдержал испытание Иов. Подумал он, что, раз уж он в Лаборатории, то это эксперимент над ним. И стал он учится ещё усердней и вскоре закрыл все двойки и тройки и продолжил жить как ни в чём не бывало. А тем временем приближалась зимняя сессия.

Иов с учителями уходят.
Снова появляются Бог с Дьяволом.

Бог — Видишь? Даже несмотря на то, что получил он плохих оценок количество несметное, смог он закрыть их и не перестал любить меня.

Дьявол — Но ты дал ему знания и обучаемость. Уберу их, и посмотрим на него.

Бог — Хорошо, я разрешаю тебе это сделать, только уверен я, что вытерпит он это и продолжит любить меня.

Бог с Дьяволом уходят

 Снова появляется Иов со стопкой пятёрками

Рассказчик — И забрал Дьявол у Иова знания по предметам и обучаемость. А сессия всё близилась. И понял Иов, что потерял он все знания. Но решил он, что это тоже испытание. И продолжил пытаться выучить материал. Но ничего у него не ..

Иов — За что мне это? Будь проклят день, когда я пошёл на вступительные экзамены

Рассказчик — ...не получалось. И завалил он сессию зимнюю. Но оставили его в лаборатории, так как обе четверти он почти всегда учился хорошо. Но говорили ему, что, так как он провалил первую сессию, то шансов остаться в школе после годовых экзаменов у него мало. Но не роптал он на Бога, а продолжал пытаться учится

Во время последних строк Иов имитирует активную умственную деятельность и что-то записывает.

выходят Бог и Дьявол.

Бог — Теперь ты просто не можешь возражать против того что это лучший ученик. Он целеустремлён и любит меня.

Дьявол — Ладно, он не сказал ничего плохого, и следовательно не попадёт ко мне в ад. Теперь, наверное можно вернуть ему знания и обучаемость.

Бог и Дьявол уходят.
Приходит Иов.

Рассказчик — И снова стал Иов учится прилежно. Зарабатывал одни пятёрки. Сдал летнюю сессию и перешёл в  следующий класс.

Во время этой речи Иов что-нибудь делает, сами придумаете
Конец.

Запасная (прошлая) версия:
текст 2

Рассказчик. В одной школе в Санкт-Петербурге учился талантливый сильный класс 9М. Был он очень умён и прилежен. Было у него 7000 пятёрок, а двоек не было вообще. Жили они очень дружно между собой и каждый день решали вместе задачи, потому что любили друг друга и никогда между собой не ссорились.
Но однажды решили преподаватели на обеде обсудить их успеваемость.
Сцена 1

Диавол. Добрый день, коллега.

Бог. У кого же вы нынче вели?

Диавол. Да вот к математикам зашёл геометрию провести.

Бог. А ведёте ли вы у 9М? Вот отличный сильный класс.

Диавол. Хорошо же им быть такими отличниками, когда вы даёте им решать задачки на 
сложение дробей. Вот послали бы вы им курсик линейной алгебры, да по теории чисел, да с зачётом, как в конце четверти. Тогда он совсем перестанет учится и скатится в двойки.

Бог. А знаете что, коллега? Возьмите себе этот класс, проведите столько спецкурсов, сколько нужно, только “не простирайте переводных экзаменов своих. Также советовать ты другим учителям можешь какие экзамены нужно провести”.

Сцена 2

(класс сидит и ничего не подозревает. Вбегает кто-то 1)

кто-то 1. Ребята, я был у расписания. Нам поставили на все дни пятую пару МАТАНА. А ещё.. а ещё.. (его прерывает следующий вошедший)

кто-то 2. Парни, подслушал я рядом с учительской, что ставят нам шестой парой спецкурс по теоретической физике. А ещё.. а ещё.. (его прерывает уже третий кто-то)

кто-то 3. Пацаны, я тут узнал, оказывается у нас на зимней сессии будет экзамены по ОБЖ, искусству и истории СПБ! нг

новая задумка: (пока без форматирования, а просто спонтанный текст)

Был в Лаборатории один ученик и звали его Иовом. Много похвал и пятёрок получал (Иов держит в руках пятёрки ) он от учителей по всем предметам. Но однажды Дьявол, поговоривши с Богом получил разрешение испытать его  нагрузкою . И велел он всем учителям заменить пятёрки на двойки и тройки (несколько “учителей” подбегают, забирают пятёрки и отдают двойки с тройками). Но ученик не сдался и закрыл все двойки и тройки и получил ещё множество пятёрок. (учителям приходится возвращать пятёрки).
И говорит Дьявол это была малая нагрузка, проверим захочет ли он уйти из ЛНМО, если отобрать у него все основные знания по предметам.
И начал получать Иов двойки, и начали подкрадываться мысли, а не уйти ли из ЛНМО. Шёл месяц, шёл другой. Пришла Зимняя сессия. За отсутствие знаний получил он 2 за все экзамены. И впал он в великую депрессию. Стал он ещё больше хотеть уйти из ЛНМО. Но настали новогодние каникулы и решил он, отдохнув, что за полгода он ещё сможет подтянуть знания и таки остаться в Лаборатории. Бог увидел, что Иов любит его и в несчастье и в депрессии. И вернул бог ему память о предметах с возможностью усваивать новый материал. И вернулись к нему все пятёрки и похвалы и накрыли они горой ту кучку двоек, полученную за первое полугодие. Прошёл он летнюю сессию и остался в ЛНМО.

ТЕКСТ III
Роли:
Иов -- Денис Ильин


Концепт: 

    Оставляем спецкурсы = испытания.
    "Убираем библейский говор." --- обсудить с режиссёром
    Убираем ясное деление Диавола и Бога.
    Развязка --- кидают кучу курсов. Потом экзамены (на класс, на учителя, неважно)    
    
    Куча вариантов, сложно выбрать.
 \end{document}