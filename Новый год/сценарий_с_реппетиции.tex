 \documentclass[10pt,a4paper]{article}
 \usepackage[utf8]{inputenc}
 \usepackage[russian]{babel}
 \usepackage[OT1]{fontenc}
 \usepackage{amsmath}
 \usepackage{amsfonts}
 \usepackage{amssymb}
 \usepackage{graphicx}
 \author{9M + 7M класс}
 \title{Сценарий со сценки}
 \usepackage{ textcomp }
 \begin{document}
 \maketitle
 Божественная столовая + звук 

Бог(Чистяков) $\Rightarrow$  Иннокентий

Дьявол  -- Анатолий 

2 посдноса из столовой.

Вошёл бог c подносоом полным здоровой крутой еды

Вошёл Дьявол с дошираком, кинул его куда-то вбок и хотел забрать еду бога. Его заметили.
%%%%%%%%%%%%%%%%%%%%%%%%%%%%%%%%%%%%%%%%%%%%%%%%%%%
Дьявол -- Как там успеваемость у вашего отличника (нужна фотка отличника с 4-х сторон с книгой в руках)

Бог -- Всё отлично, его оценки как всегда высоки.

Дьявол -- Видно вы ему задания не самые сложные даёте. Ну не может же он быть таким умным.

Бог -- Да вы что? Иов -- лучший из лучших, умнейший из умнейших.

Дьявол -- Я уверен, что он не пройдёт моё испытание.

Бог -- Хотите испытать его?

Дьявол -- Давайте заключим пари. Победившему безлимтный абонемент в божественную столовую. Что же касается проигравшег(о).. (пауза, на экране картинка доширака)

Бог -- я-то в Иове уверен, так что по рукам.
%%%%%%%%%%%%%%%%%%%%%%%%%%%%%%%%%%%%%%%%%%%%%%%%%%%

Пожали руки $\Rightarrow$ звук грома

Пошли фотки ученика самого бога. 

ритмическая музыка

атрибут лучшего ученика

Выходит с пятёрками в шапке профессора.

Какой-то диалог ученика и Дьявола

Маска -- фотография Чистякова

Учитель вы как-то изменились (может быть аля красная шапочка)

Первое испытание -- задача (формулировка сложной задачи -- коламбур)

Второе испытание -- какой-то предмет, который не по нпшему профилю

Третье испвтние-барышня

Финал -- Повтор первой мезо-сценный (перевёртыш)


 \end{document}