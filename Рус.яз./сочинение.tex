\documentclass[12pt,a4paper]{article}
 \usepackage[utf8]{inputenc}
 \usepackage[russian]{babel}
 \usepackage[OT1]{fontenc}
 \usepackage{amsmath}
 \usepackage{amsfonts}
 \usepackage{amssymb}
 \usepackage{graphicx}
 \usepackage{setspace}
 \doublespacing
 \author{Коченюк Анатолий}
 \title{Сочинение по цитате}
 \begin{document}
 \maketitle
 
 Цитата: Обладая и лексическим, и грамматическим значением, слово способно объединяться с другими словами, включаться в предложение.
 \newline
 
 Данную цитату я понимаю так: слово, имея грамматическое и синтаксическое значение, соединяясь такими же словами, может образовывать предложения. Приведу пару примеров.
 
 Для начала рассмотрим предложение ''Зелёный кот прыгнул в окно.'' с точки зрения грамматики. ''Кот'' является подлежащим, а "прыгнул" сказуемым. Вместе они образуют грамматическую основу, то есть именно они передают основной смысл этого предложения. Также в нём есть второстепенные члены предложения: ''зелёный'' -- это определение, которое говорит какой именно кот совершает действие, сочетание слов ''в окно'' -- это обстоятельство, которое более подробно описывает действие. 
 
Теперь рассмотрим это предложение с точки зрения лексики. Рассмотрим у каждого слова его лексическое значение, если оно есть. Первое слово ''Зелёный'' описывает характеристику (в данном случае цвет) чего-либо. Далее идёт слово ''кот''. Оно означает животное и именно к нему относится предыдущее слово. ''Прыгнул'' означает действие перемещения чего-либо вверх в пространстве.
Ну и сочетание слов ''в окно'' поясняет направление действия, описанного предыдущим словом. 
 
 Учитывая оба значения каждого из слов, можно соединить (включить) их в единое предложение, тем самым показав истинность цитаты. 
 
 \end{document} 