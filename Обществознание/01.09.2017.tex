 \documentclass[12pt,a4paper]{book}
 \usepackage[utf8]{inputenc}
 \usepackage[russian]{babel}
 \usepackage[OT1]{fontenc}
 \usepackage{amsmath}
 \usepackage{amsfonts}
 \usepackage{amssymb}
 \usepackage{graphicx}
 \usepackage[left=2cm,right=2cm,top=2cm,bottom=2cm]{geometry}
 \author{Анатолий Коченюк}
 \title{Конспект по обществозанию}
 \begin{document}
\maketitle 
\newcommand{\hw}{\Huge}
\chapter{Экономика (продолжение)}

\section{Государственный бюджет}
Определение --- смета доходов и расходов государства на определённый период времени, составленное с указанием источников поступления государственных доходов и направлений расходования денег.\\
Составляется правительством и утверждается государственной думмой.\\
По окончании инансового года правительство РФ должна отсчитаться об исполнении бюджета\\.\\
Бюджет представляет себя бюджетной системой. Она основана на экономических отношениях и государственном устройстве РФ. Регулируема нормами права. \\
А также совокупность федерального бюджета, бюджета субъектов РФ, местных бюджетов и государственных небюджетных фондов.\\
Всю эту систему можно разделить на 3 части :


\begin{tabular}{c||c|c}
\hline
Уровень бюджета &
\begin{tabular}{c}
Составляют бюджет\\
Органы bсполнительной власти\\
\end{tabular} &
\begin{tabular}{c}
Утверждают бюджет\\
органы Заканодательной власти\\
\end{tabular}\\
 
\hline\hline
1. Федеральный & Правительство РФ  & Государственная думма \\
&&-->и Федеральное собрание\\
\hline 
2. Бюджеты субъектов РФ & Правительство (Администрация) субъектов&
\begin{tabular}{c}
 Государственные собрания,\\
  Законодательные собрания,\\
  Областные, Краевые думы,\\
   Парламент\\
\end{tabular}\\
\hline
3. Местные (Сельские,& Админимстрации во главе с  &Городские собрания,  \\
  Городские, Районные)&мэрами, председателями &районные советы, думы..\\
\hline

\end{tabular}
\\.\\.\\.\\.\\.\\.\\.\\
\begin{tabular}{l||l}
Расходная & Доходная\\
\hline\hline
Социальные нужды (около 50 \% )& Налоги\\
Поддержание оборонспособности страны(15-20\%) & Государственные займы\\
Обслуживание государтсвенного долга &--> (также может занимать у населения)\\
Предоставление дотаций(помощь) предпринимателям  & напечатать деньги(эмиссия)\\
-->развитие и обслуживание инфраструктуры& Займы у международных организаций\\
-->(дороги, электрообслуживание, ЖКХ..)& \\
 Структура расходной части зависит от поставленных &\\
 задач перед государством& \\
\end{tabular}
\\.\\.\\.\\.\\
Бюджетная политика включает в себя соотношение между доходной и расходной частями государственного бюджета\\
Существует 3 таких части :
\begin{enumerate}
	\item Сбалансированный -- расходы $\approx$  доходам
	\item Дефицит -- расходы $>$ доходы
	\item Профицит -- расходы $<$ доходы
\end{enumerate}

Источники покрытия бюджетного дефицита :
\begin{itemize}
	\item Привлечение средств, посредством фирм и др. организаций(растёт гос. долг) 
	\item Использование кредитов ЦБ(центрального банка). ЦБ печатает деньги и отдаёт их осударству, а потом возвращает их обратно.
	
\end{itemize}
Факторы влияющие на состояние государственного бюджета:
\begin{enumerate}
	\item Долгосрочные тенденции налоговых поступлений и государственных расходов в стране
	\item Фазы экономического цикла : 
	\begin{itemize}
		\item Спад / кризис --- дефецит
		\item Подъём / пик --- профицит / сбалансированный
	\end{itemize}
	\item Текущая политика государства
\end{enumerate}
\subsection{Государственный долг}

Государственный долг --- сумма задолжности государтсва по выпущенным (ценные бумаги) и не погашенным  займам, включая начисленные по ним проценты.\\
Обычно, чем больше и развитие государство, тем больше долг.\\
Государственный долг :
\begin{itemize}
	\item Внутренний -- долговые обязательства правительства перед юридическими и физическими лицами, выраженный в национальной валюте.\\
	Долговые обязвтельства бывают двух видов :
	\begin{itemize}
		\item Рыночные -- долговые обязательства, выпущенные государством на внутреннем рынке ввиде ценных бумаг, как правило облигаций.
		\item Нерыночные -- возникают в результате исполнения бюджета(бюджет выполняется в рамках прописанных условий, составляется финансовая программа и могут возникнуть непредвиденные затраты )
	\end{itemize}
	\item Внешний -- задолжность государства по непогашенным внешним займам  и не выплаченным по ним процентам международным организациям и иностранным банкам, правительством  ... и выражен в иностранной валюте.
\end{itemize}
Причины возникновения внутреннего государтсвенного долга :
\begin{enumerate}
	\item Получение государтсвом кредитов от коммерческих банков, юридических лиц, выраженных в национальной валюте
	\item Осуществление государством внктренних займов
	\item Предоставление бюджетных ссуд одним уровнем бюджетной системы другому
\end{enumerate}
Важны не абсолютные показатели долга, а его отношения к другим экономическим показателям :
\begin{itemize}
	\item Долг : ВВП $\ngeqslant$ 80 \%
	\item  Велечина долга на душу населения
	\item  Долг : объём экспорта $ \ngeqslant$ 200\%
	\item  Долг : ЗолотоВалютному резерву )
	\item Расходы по обслуживанию долга : сумм экспорта $\ngeqslant$ 15-20\%
\end{itemize}
Реструкоризация долга -- процесс, при котором пересматриваются условия обслуживания долга (проценты, дата возврата).\\
Меры по управлению государственного долга :
\begin{enumerate}
	\item Не допущение долговой ловушки, при которой все ресурсы работают на выплаты долга а не на увеличение национального богатства.
	\item Нейтрализация негативных последствий государтсвенного долга.
	\item Эффективное использование средств, заимствование, т.е. направление их в проекты, котрые в отведённый срок позволит обеспечить доходы, превышающие размер долга и проценты по нему.
	\item Поиск средств для выплаты долга
\end{enumerate}

{\Huge (Д/з -- проверочная по государственному бюджету.)}
\section{Мировая Экономика}
Мировая Эконмика -- противоречивая целостность национальных экономик, связанных между собой международными экономическими отношениями, ан основе международного разделения труда.\\
Страны делятся на 3 категории :
\begin{enumerate}
	\item 1-ый эшэлон -- развитые страны с рыночной-смешаной экономикой
	\item Страны с  развивающимися экономикой
	\item Старны с переходным типом экономики
\end{enumerate}
Предпосылки международной специализации государтсва :
\begin{enumerate}
	\item Природные условия страны : климат, географическое положени, наличие полезных ископаемых и других ценных ресурсов
	\item Уровень экономического и научно-технического развития\\
	\emph{Страны с развитой эеонмикой специализируюися на производстве готовых продуктов, а страны с развивающейся экономикой производят сырьё}
	\item Сложившиеся традиции в производстве
\end{enumerate}
Муждународное Разделение Труда (МРТ) --- устойчивое  производство товаров и услуг в отдельных странах , сверх внутренних потребностей, в расчёте на мировой/международный рынок.\\.\\
Межународные экономические отношения (МЭО) --- комплекс торговых, производственных, научно-технических и финансовых связей между государствами, приводящих к обмену ресурсами и мовместной экономической деятельности.\\
Формы МЭО :
\begin{enumerate}
	\item Медународная торговля товарами и услугами 
	\item Движение капиталов и зарубжных инвестиций
	\item Миграция рабочей силы 
	\item Валютно-кредитные отношения
	\item Обмен в области научно-технических открытийб обмен в области науки и техники
	\item Межотрослевая кооперация производства
\end{enumerate}
\section{Международная торговля}
регулируема всемирной торговой организацие (ВТО). Следит за соблюдением норм свободной торговли.
3 основных принципов работы ВТО:
\begin{enumerate}
	\item Торговый режим должен быть одинаков для всех партнёров
	\item Постепенное снижение внешнеторговых пошлин 
	\item Устойчивость и предсказуемость в внешнеторговой политике
\end{enumerate}

Импорт --- закупка и ввоз товаров и услуг (нанимание) из других стран для продажи на внутреннем рынке.

Экспорт -- продажа товаров и услуг за границу для реализации на Мировом Рынке

Разница между Экспортом и Импортом -- сальда.

Эмбарбо -- запрет на ввоз (импорт) товаров. Может касаться либо целой страны либо отдельного товара

Экспорт сырья -- занимаются страны с развивающейся экономикой
\section{Внешнеторговая политика государства}
\begin{itemize}
	\item [I] Протекционизм. От слова защита/защищать . Политика государства, направленная на защиту интересов внутренних производителей от иностранных конкурентов.\\
	Формы:
	\begin{itemize}
		\item Селективная -- направлен против отдельных стран или товаров
		\item Отрослевая -- защищает отдельные отросли, как правило сельское хозяйтсво
		\item Коллективный -- Производится объединениями стран, против других стран, не входящих в них.
		\item Скрытный -- производится ч помощью методов внутренней экономической политики. (финансирование конкурирубщей отросли из бюджета ...)
	\end{itemize}
	\item [II] Свободная торговля :: Политика государтсва, ориентированная на развитие свободной международной торговли.
	\item [III] Умеренная :: сочетает в себе элементы протекционизма и свободной торговли
\end{itemize}

Страны с развивающейся (переходящей) экономикой придерживаются протекционизма.

Страны с хорошим уровнем производства 

Рынок:
\begin{itemize}
	\item Мировой
	\begin{itemize}
		\item Тарифный -- 
		\begin{itemize} 
			\item тарифные/таможенный планы на импортируемые товары, 
			\item таможенные союзы
			\item экспортные тарифы
		\end{itemize}  
		\item Нетарифный --
		\begin{itemize}
			\item Установление квот (введение количественного ограничение на тот или иной товар)
			\item  Установление стандартов на продукцию
			\item Эмбарго
		\end{itemize}
	\end{itemize}
	\item Нацинальный -- релугируется внутренней экономической политикой
\end{itemize}

Экономическая интеграция -- процесс, создания региональных хозяйственных коллективов, на основе международного регулирования внешней торговли и движения факторов производства.

Типы экономической интеграции:
\begin{itemize}
	\item Зоны свободной торговли. Предполагает взаимную отмену торговых пошлин между участников интеграционной группировки, но сохранение у каждой из этих сторон особой внешнеторговой политики по отношени к третьим странам
	\item предполагают снижение таможенных пошллин на товары внешнеторгоаого партнёра, по сравнению с уровнем, которыми облагаются товары третьих стран
	\item Таможенные союзы -- предполашают взаимную отмену таможенных пошлин и унификацию внешнеторговой политики, относительно третьих стран.
	\item Общий рынок -- в его рамках на ряду со свободной торговлей обеспечивается либерализиции перетекания капиталов и рабочей силы между странами, между странами осуществляется согласование экономисеской политики.5
	\item Экономические союзы -- происходит унификации экономической, социально политики, социальной политики, международной политики, научно-технического прогресса. Между странами участниками формируется единая (общая) валюта
	\item Валютные союзы -- происходит согласование национальных валют, устанавливаются по сошлашению фиксированные национально-валютные курсы, которые целенаправленно поддерживаются Центральным банками стран - участниц, создаётся единая региональная валюта с единым эмиссионным центром
	\item Полная экономичесская интеграция -- Производится полная унификация экономической политики, формируется общее законодательство и создаётся единая экономическая база
\end{itemize}
\section{Международная валютно-финансовая система}
-- Закреплённая международными договорами форма организации валютно-финансовых отношений, функционирующих самостоятельно или обсуживающих международное движение товаров и факторов производства.
Структура международной валютно-финансовой системы  по сфере деятельности:
\begin{enumerate}
	\item Валютное отношение по поводу условий обращения и взаимной конвертации национальных валют, механизмов регулирования валютных курсов  
	\item Финансово отношение , которое охватывает международные финансовые рынки торговли ценными бумагами, валютой и кредитованием (акуии междунородных компаний)
	\item Экономически отношения, складывающиеся в процессе международных расчётов платёжных балансов различнсых стран
\end{enumerate}
По сфере отношений :
\begin{itemize}
	\item Международно-валютный рынок -- сфера экономических отношений, где осуществляются операции по покупке, продаже золота иностранной валюты и платёжных документов в иностранной валюте 
		\item Международно-финансовый рынок -- рынок кратких, средних и долгосрочных кредитов, инвестиций, ценных бумаг, вкаладов ... , функционоирующих на международном уровне.	
	\begin{enumerate}
		\item Рынок капитала -- средне и долго срочные отношения (ценные бумаги, инвестиции)
		\item Денежный рынок -- краткосрочне отношения (кредиты, инвестиции, ценные бумаги)
	\end{enumerate}
\end{itemize}
Деятельностью по регулированию международно-валютной финансовой системы занимается меджународный валютный фонд, организации экономического сотруднечества и развития, международный банк реконструкции и развития.

{\hw ДЗ -- проверочная по международной экономике}
\chapter{Человек и Общество}
 \section{Природное и общественное в человеке}
 Существует 3 концепции происхождения человека:
 \begin{enumerate}
 	\item Религиозная, креационистская 
 	\item Теория палеовизита -- человек имеет внеземное происхождение
 	\item Естественно-научные концепции:
 	\begin{itemize}
 		\item Эволюционная (Теория Чарльза Дарвина) -- объеснение -- мутации
 		\item Теория Энгельса -- общество и особое отношение внутри группы приматов 
 	\end{itemize}
 \end{enumerate}
 
 Различия между животным и человеком:
 \begin{itemize}
 	\item Биологические 
 	\begin{enumerate}
 		\item Человек -- прямоходящий
 		\item Наличие членораздельной речи
 		\item Отставленный большой палец на ладони
 	\end{enumerate}
 	\item Социально-Психологичекие
 	\begin{enumerate}
 		\item Человек обладает способностью к сознательной целенаправленной творческой деятельности. Выражается в способности прогнозирование последствий своего поступка. Человек на пртяжени своей жизни занимает разные доожности, роль в обществе не закреплена.
 		\item Способность к изготовлению и испльзованию орудий труда. Определённые животные используют палки и камни в своей жизнедеятельности, но человек использует неоднократно своё орудие труда, а также развивает и улучшает его.
 		\item Челоек преобразует окружающюю действительность под свои 
потребности
		\item Человек имеет не только биологические, но и социально-духовные потребности
 	\end{enumerate}
 \end{itemize}

\textbf{Индивид} --- еденичный представитель человеческого рода, конкретный носитель всех социальных и психологических черт человечества

\textbf{Индивидуальность} --- неповторимое своеобразие проявлений человека, подчёркивающее исключительность проявлений человека, многосторонность деятельности человека

\textbf{Личность} --- человеческий индивид, являющийся субъектом сознательной деятельности, обладающий совокупностью социально значимых черт, свойств и качеств, которые он реальзует в общественной жизни.

Структура Личности:
\begin{enumerate}
	\item Социальный статус --- место человека в системе общественных отношений
	\item Социальная роль --- образ поведения, одобренный нормативно и соответсвующий социальному статусу
	\item Направленность -- потребности, интересы, взгляды, убеждения, мотивы поведения человкека
\end{enumerate}
\section{Мировоззрение и его формы}
Определяет поступки, убеждения и т.д.
Структура внутреннего или духовного мира человека:
\begin{enumerate}
	\item Познание или интеллект. Потребность в знании о себе, окружающем мире, смысле и назначении жизни и т.д. Потребность в знании формирует интеллект. Интеллект -- совокупность умственных способностей, прежде всего способность получать и усваивать новую информацию на основе той, которая уж имеется.
	\item Эмоции --- субъективные переживания по поводу ситуаций, явлений и окружающей действительности.
	\item Чувства --- эмоциональные состояние, более длительные, чем эмоции, и имееют чётко выраженный предметный характер.
	\begin{itemize}
		\item Нравственные: дружба, любовь
		\item Эстетические: отвращение, восторг, тоска
		\item Интеллектуальные: любопытсво, сомнение, любознательность
	\end{itemize}
	\item Мировоззрение -- система взглядов, понятий и представлений об окружающем мире, которая обуславливает направленность личности
	\item Также отдельно можно выделить память, ведь она делает нас отдельным человеком.	-
\end{enumerate}

Структура Мировоззрения:
\begin{enumerate}
	\item Мироощущение -- эмоционально-психологическая сторона мировоззрения (чувства, настроение и т.д)
	\item Мировосприятие -- образ мира в наглядных представлениях
	\item Миропонимание -- чувства, которые доминируют в вас, ваше понимание окружающего мира. Познавательно-интеллектуальная сторона мироввоозрения
\end{enumerate}
Особенности мировоззрения:
\begin{enumerate}
	\item мировоззрение всегда исторично. Всегда связано с переживаемой обществом стадией развития, отражает те проблемы, которые наиболее актуальны в данном обществе.
	\item В процессе формирования мировоззрения могут проявляться следующие точки зрения:
	\begin{itemize}
		\item Догматизм (мнение с латинского) --- форма мышления и действия, харктеризующаяся костностью, неподвижностью, стреотипностью, стремлением к авторитарности 
		\item Скептицизм -- философское напраление, использующее сомнение в познавательных возможностях мышления.
		\item Умеренный критицизм -- философское направление, которое считает необходимым предпослать познанию вещей исследование о возможностях самого познания.
	\end{itemize}
	\item Мировоззрение всегда связано с убеждениями, т.е. с устойчивыми взглядами на мир, идеалами и принципами, стремление воплотить их в жизнь, через свои постуки и действия.
	\item Носителем может быть как отдельный человек, так и целое общество
\end{enumerate}
Два пути формирования мировоззрение:
\begin{enumerate}
	\item Стихийная. На основе обыденного опыта, под влиянием жизненных условий.
	\item Осознанная. Посредством целенаправленной теоритической разработки фундаментальных принципов, идей и идеалов.
\end{enumerate}
Исторические типы мировоззрения:
\begin{enumerate}
	\item Мифологическое. основана на эмоционально-образном и фантастическом отношении к миру. Эмоциональный компонент превалирует (доминирует) над разумными объяснениями. Мифология происходит прежде всего из потребности объяснить непонятные явления природы и окружающей действительности.
	\item Религиозное. Религия --- система верований и практик, направленное на почитание священных объектов, обособленных и запретных, верований и практик, объединящих в одно сообщество всех тех, кто их разделяет (по Эмилю Дюркгейну).
	Свойственен Догматизм (неотступное следование религиозных норм). Хорошо разработанная система моральных приницпов, правил, законов). Религия определяет поведение человека в большинстве жизненных обсоятельств.
	\item Философский или научный. Системность, последовательность, логичность суждений. Присутсвует высокая степень обобщения. Основное отличие от Мифологического и религиозного мировоззрений является опора на рациональную сторону познания. Исторически выражается присутствием свободы мысли.
\end{enumerate}
Разновидности личностного мировоззрения:
\begin{enumerate}
	\item Обыденное или житейское. Формируется преимущественно стихийно. Основная деятельность направлена на удовлетворение потребностей
	\item Религиозная. Связана признанием сверхъестественного начала
	\item Философское. Связано с получением обобщённых знаний об окружающей дествительности, установлением норм, ценностей и идеалов, определяющих средства, цели и характер деятельности человека.
	\item Научное. Теоретическое и практическое осмысление результатов научной деятельности. Ориентирование в повседневной жизни на достижение науки.
	\item *Гуманистическое (отчасти и философское и научное) основано на признании ценности каждой человеческой жизни.
\end{enumerate}
Роль мировоззрения в жизни человека.
\begin{enumerate}
	\item Даёт человеку ориентиры и цели для его практической и теоретической деятельности.
	\item Позволяет понять как лучше и правильнее достичь намеченных ориентиров и целей, предоставляет методы познания и устанавливает форму деятельности.
	\item Даёт возможность определить истинные ценности в жизни и обществе.
\end{enumerate}
Менталитет -- совокупность всех итогов познания, оценка их на основе предшествующей культуры и практической деятельсти национального сознаниям. 
\section{Виды Знаний}
Познание -- процесс деятельности человека, основным содержанием которого является отражение обкружающей действительности. Субъектом познания может быть что-угодно, влоть до самого познания.

Агностицизм (от гр. гнозис -- ) -- частично или полностью отрицает возможность  познания, познание не даёт объективных знаний о действительности. Представители : Кант, Спенсор, Беркли

Скептецизм -- не отрицает приницппиальной возможности познания, но выражает сомнения в том, что знания о мире носят достоверный характер.

Оптимизм -- утверждает приницпиальную возможность достоверного познания мира

Виды познания:
\begin{enumerate}
	\item Научное. Результат -- научное знание
	\item Художественное. Результат -- ходожественное знание
	\item Обыденное. Результат -- обыденное знание
	\item Религиозное
	\item Социальное
	\item Мифологическое
\end{enumerate}
Ни один из них нее является изолированым. Все взаимосвязаны.
2 уровня познания:
\begin{itemize}
	\item Чувственныый -- познаём окружающюю действительность с помощью органов чувств. Выражается в следущих формах:
	\begin{itemize}
		\item Ощущение -- отражение отдельных свойств предмета (явления, процесса), возникающих в результате их непосредственногго воздействия на органы чувств
		\item Восприятие -- чувственный обаз целостной картины предмета, процесса или явления, непосредственно воздействующих на органы чувств
		\item Предствавление -- чувственно-наглядный обобщённый образ объекта, не воздействующего на органы чувств
	\end{itemize}
	\item Рациональный -- является более сложным спосом отражения окружающей действительности, который осуществляется посредством мышления. Формы:
	\begin{itemize}
		\item Понятие -- мысль, утверждающая общие и существенные свойства предмета, явления или процесса
		\item Суждение -- мысль, утверждающая или отрицающяя что-либо о предмете, процессе или явлении.
		\item Умозаключение -- мысленная связь нескольких суждений и выделение из них нового суждения.
	\end{itemize}
\end{itemize}
Эмперизм -- единственный достоверный источник познания это чувства. сторонники : Томас Гопс, Джон Локк, Беркель, Девид Юм, Фрэнсис Бэкон.\\
Рационализм -- знания могут быть получены с помощью ума без опоры на чувства. Сторонники: Спиноза, Лейбниц, Рено Декарт, Гегель.
Особенности чувственного:
\begin{itemize}
	\item Непосредственность, выражающаяся в прямом воспроизведении объекта
	\item Наглядность и предметность, возникающая в результате познания
	\item Воспроизведение внешних свойств и сторон познания
\end{itemize}

особенности рационального познания:
\begin{itemize}
	\item Абстрактность и обобщённость, возникающая
	\item Воспроизведение объектов на основе внутренних закономерных связей и отношений
	\item Опора на чувственно-воспринимаемый опыт
\end{itemize}

типы Интуиции :
\begin{itemize}
	\item Мистическая. Связано с практической деятельностью человека
	\item Интелектуальная. Связана с умственной деятельностью
\end{itemize}

Отличительной чертой интуиции является получение достоверного знания без осозная процесса познания.

В научной среде избегают такого способа познания.

Знание -- результат познания окружающей действительности, содержание сознания, полученное в ходе активного отражения идельного воспроизведения объективных закономерных связей и отношений окружающего мира.

Знание -- отношение человека к окружающей действительности.\footnote{Самое широкое определение. Другие тоже надо иметь ввиду}
Основные виды знаний:
\begin{itemize}

	\item Самое распространённое знание -- житейское.
	Оно носит эмперический характер и базируется на самом базовом уровне познания. Руководствуется на зравом смысле.
	\item Практическое. Связано с какой-то деятельностью, с преобразованием окружающей среды. В первую очередь является знанием о деятельности
	\item Художественное. Строится на образном представлении человека об окружабщем мире.
	\item Научное. Достоверное обобщение фактов. Осуществляется в виде оформленных понятий.
	\item Линостное. Субъективная особенность окружающей действительности, зависящее от от особенностей человека.
	\item Рациональное. Результат мыслительной деятельности
	\item Иррациональное. Интуиция
\end{itemize}

\subsection{Научное познание}
Виды вненаучного познания:
\begin{itemize}
	\item Донаучное знание. Прототип, предпосылки к научному знанию
	\item Паранаучное и лженаучное знание. Астрология -- паранаучное знание.
	\item Антинаучное. Связано с какими-то утопиями, с сознательным искажением современного знания
\end{itemize}

Начное знание -- особый вид познавательной деятельности, направленный на выработку обоснованных системно-организованных им объективных знаний о природе, человеке и, в некоторых случаях, самого познания.

Особенности научного познания:
\begin{enumerate}
	\item Объективность полученных знаний
	\item Проверяемость.
	\item Развитость понятийного аппарата.
	\item Универсальность
	\item Использование специальных способов и методов познания.
\end{enumerate}

Уровни научного познания:
\begin{enumerate}
	\item Имперический. Выявление объективных фактов со сторон их очевидных связей. 
	\item Теоретический. Выявление фундаметальных закономерностей, обнаружение скрытых внутренних знаний
\end{enumerate}

Форму научного познания:
\begin{enumerate}
	\item Научный факт. Отражение какого-то объективного факта в человеческом сознании и описание его посредством некоторого языка
	\item Имперический закон. Объективная существенная повтряющаяся и устойчивая между какими-то процессами, явлениями и т.д.
\end{enumerate}

Формы теоретического познания:
\begin{enumerate}
	\item Проблема -- осознанная формулировка вопросов, возникающих в процессе познания и требующая ответа. Может быть теоретической или практической.
	\item Гипотеза -- научное предположение, свормулированное н аотснове ряда факта, истинное значения которого не определено.  
	\item Теория -- наиболее развитая область научного познания, дающая целостное отображение закономерных и существенных связей определённой области. действительности.
\end{enumerate}
Струтура теории:
\begin{itemize}
	\item Исходные основания -- факты, аксиомы, понятия, принципы, законы...
	\item Идеалтзироаванный объект -- цель нашей теории.
	\item Логика методология (совокупность методов, используемых в данной теории), применяемая для построения теории.
	\item Совокупность законов и утверждений, выведенных из данной теории.
\end{itemize}

Методы имперического познания:
\begin{itemize}
	\item Наблюдение
	\item Измерение
	\item Эксперимент
	\item Классификация
	\item Описание
	\item Сравнение
	\item Систематизация
\end{itemize}

Методы научного познания
\begin{enumerate}
	\item Единственство исторического и логического методов
	\item Формализация 
	\item Метематизация
	\item Восхождение от конкретного к абстрактному или от абстрактного к объектному.
\end{enumerate}
Универсальные методы познания:
\begin{enumerate}
\item анализ
\item синтез
\item индукция
\item дедукция
\item моделирование
\item астракция
\item идеализация
\end{enumerate}


\section{Истинность и её критерии}

Несколько определений истины:
\begin{enumerate}
	\item Соответствие знаний (полученной информации) действительности.
	\item То, что подтверждается опытом.
	\item Некая конвенкция или соглашение.
	\item Свойство самосогласованности знаний. (когда знание не противоречит себе)
	\item Полезность полученных знаний на практике. (то, что упрощает или приносит пользу человеку в его существовании)
\end{enumerate}
Дополнительные точки зрения на получаемые знания:

Империзм -- всё знание об окружающем мире обусловлено нашим опытом. [Фрэнсис Бэкон]

Сенсуализм -- все знание об окружающем мире получаются только при помощи ощущений.\\
Истинность бывает абсолютной и относительной:
\begin{itemize}
	\item Абсолютная -- знания, которые на 100\% верные.
	\item Относительная -- знания, зависящие от определённых условий (места, времени..) их получения. Неполное, неточное знание, которое соответсыует какому-то уровню познания.
	\item [*] Объективная -- знание, которое не зависит ни от человека, ни от человечества в целом.
\end{itemize}
Возможныекритерии истинности:
\begin{itemize}
	\item Соответствие законам логики (тождество, непротиворечие и закон исключенного третьего). 
	\item Соответствие ранее открытым законам науки. 
	\item Соответствие фундаментальным законам науки.
	\item Практика или полезность. 
	\item [*] Простота или экономичность формулировок формул и т.д.
\end{itemize}


\section{Мышление и деятельность}
Мышление -- активный процесс отражения объективного (окружающего) мира в понятиях, суждениях, теориях, терминах и т.д.

Компаненты мышления:
\begin{itemize}
	\item Биологическая основа мышления -- головной мозг человека
	\item Средство мышления -- язык
\end{itemize}
Формы выражения мысли:
\begin{itemize}
	\item вербальные и речевые
	\item ассоциативно-образные
	\item понятийные и категориальные
	\item деятельносто-инструментые или деятельно-орудийные
\end{itemize}

Основа мышления -- чувственный опыт

Этапы Мышления:
\begin{enumerate}
	\item Формирование задачи (вопроса), определение предмета мышления и направленности мыслительного процесса
	\item Решение поставленной задачи (анализ, синтез, сравнение, классификация, описание, обобщение, абстагирование, конкретизация)
	\item Достижение нового знание, которым раньше вы не владели
\end{enumerate}

Мышление носит личностный характер -- каждый человек испытавает различные эмоции, разные проблемы привлекают разных дюдей, у различных людей различные мотивы.

Типы мышления:
\begin{enumerate}
	\item Образное мышление
	\item Понятийное и теоретическое мышление -- задача может быть изложена теоретическим способом
	\item Знаковое мышление -- знания существуют в языковых знаках
\end{enumerate}

Деятельность -- способ отношения человека к окружающему миру, состоящий из: преобразовании, подчинении, приспособлениии к своим целям.

Основное различие нас и животных -- орудия труда

Компоненты деятельности:
\begin{itemize}
	\item Субъект(тот, кто осуществляет деятельность) и объект(то, на что направлена деятельность)
	\item Мотив
	\item Цель
	\item Методы и средства
	\item Процесс
	\item Продукт
\end{itemize}

Мотив -- совокупность внутренних и внешних условий, выхывающих активность субъекта и определяющих направленность деятельности.

Основные мотивирующие факторы с точки зрения Бихевиаризма положений
\begin{enumerate}
	\item Успех -- Чем чаще вознаграждается какое-то действие, тем большее количество раз человек хочет его повторить
	\item Стимул -- Если какое-то действие привело к успеху, то человек хочет его повторить большее количество раз
	\item Цельность -- Чем более ценно для человека результат, тем большее количество раз человек хочет его повторить
	\item Насыщение - голодание -- Чем чаще человек получал какое-то особое вознаграждение, тем меньше оно становится ему нужно
	\item Агрессия - одобрение -- Если человек ожидает и не получает вознаграждение, то он захочет показать свою агрессию, а если размер награды превышает ожидаемый, то он бует проявлять одобрение
\end{enumerate}

Цель -- осознанный образ того результата, на достижение которого направлена деятельность человека
Виды деятельности по Максу Вэберу:
\begin{enumerate}
	\item Целерациональная деятельность -- рационально продуманная и поставленная цель. Эффективно действует тот индивид, который 
	\item Ценностнорациональная деятельность -- 
	\item Традиционная -- Часто автоматическая реакция. Самый популярный вид
	\item Аффиктивное поведение -- Обусловлено эмоциональным состоянием человека. Он немедленно хочет удовлетворить свои потребности.
\end{enumerate}

\section{Особые виды деятельности}
\subsection{Игра}

Игра -- особый вид деятельности, целью которого является сам процесс, а не производство какого-либо продукта
	
Особенности игры:
\begin{itemize}
	\item Игра происходит по определённым правилам
	\item В определённой ситуации
	\item используются замещающие объеты
	\item Влияет на развитие личности
\end{itemize}

\subsection{Общение}
Общение -- особый вид деятельности, при котором происходит обмен результата умственной деятельности между двумя или несколькими учатниками

Общее между общением и коммуникацией -- обмен информацией
Различия общения и коммуникации:
\begin{itemize}
	\item Получатель информации в общении -- человек. В коммуникации получателем может быть как человек, так и машина
	\item Общение подразумевает два субъекта (человека). Коммуникация -- однонаправленный поток
\end{itemize}

Структура общения:
\begin{enumerate}
	\item Субъекты -- люди, группы людей, 
	\item Цель 
	\item Содержание (идеи, эмоции, чувства)
	\item Средства
\end{enumerate}

Классификации общения:
\begin{enumerate}
	\item По используемым средствам:
	\begin{enumerate}
		\item непосредственная -- лицом к лицу
		\item опосредованое -- с помощью срециальных средсв (газет, писем) 
		\item Прямое -- личный контакт
		\item Косвенное -- через посредника
	\end{enumerate}
	\item По субъектам общения
	\begin{enumerate}
		\item Между реальными субъектами
		\item Между реальным субъектом и иллюзорным партнёром (игрушка, животное)
		\item Между реальным субъектом и воображаемым партнёром (его нет)
		\item Между воображаемыми партнёрами
	\end{enumerate}
\end{enumerate}
Функции общения:
\begin{enumerate}
	\item Социализация -- формирование и развитие межличностных отношений способствует становлению человека как личности
	\item Психологическая -- осуществлением определённого воздействия на психическое состояние человека
	\item Познавательная / Информативная -- познание людьми друг друга и окружающего мира
	\item Отождествления или противопоставления (выражение причатсноти человека к той или иной социальной группе)
	\item Организационная -- организация совместного действия людей 
\end{enumerate}
\begin{enumerate}
\item общение = деятельность
\item общение -- вид дейтельности
\item общение и деятельность разные вещи
\end{enumerate}

\subsection{Обучение}

Обучение -- вид деятелности, целью которой является приобретение человеком новых знаний, умений, навыков

\begin{enumerate}
	\item Орагнизованное (курсы, учреждения)
	\item Неорганизованное 
\end{enumerate}
\subsection{Труд}

Труд -- вид деятельности, который направлен на достижение практического результата

Особенности труда
\begin{enumerate}
	\item Целесообразность
	\item Нацеленность на достижение запрограммированных и ожидаемых результатов
	\item Наличие мастерства, умения, навыков
	\item Практическая полезность
	\item Наличие результата
	\item Преобразование внешней среды или среды обитания человека
	\item Развитие личности или приобретение опыта
\end{enumerate}

\subsection{Творчетво}
Творчетво -- собый вид деятельности, результатом которого является что-то новое, никогда ранее не существовавшее

Взгляды:
\begin{itemize}
	\item Творчество является компонентом любого вида деятельности 
	\item Творчество -- самостоятельный вид деятельности
\end{itemize}

Считается, что каждый человек в той или иной мере обладает способностью к творчеству. Однако, они могут развиваться или пропадать.

Механизмы творческой деятельности:
\begin{enumerate}
	\item Комбинирование или варьрование уже имеющихся знаний
	\item Воображение
	\item Фантазмя (характеризуется особой яркостью новых представлений в сознании)
	\item Интуиция -- получение нового знания без осознания процесса познания
\end{enumerate}

\section{Основные классификации деятельности}
По объектам и результатам:
\begin{enumerate}
	\item Материальная (практическая) -- связана с созданием необходимых для удовлетворения потребностей человека и общества вещей, материальных ценностей, экономических благ.
	\begin{itemize}
		\item Материально-производственная -- преобразование природы
		\item Социально-образовательная -- преобразование общества
	\end{itemize}
	\item Духовная -- связана с созданием идей, образов, научных, художественных и нравственных ценностей 
	\begin{itemize}
		\item Познавательная -- отражение окружающей действительности в сознании
		\item Ценностно-ориентировочная -- связана с положительным или отрицательным отношением к явлениям окружающей дествительности и формирование через это мировоззрения
		\item Прогностическую -- связана с планированием изменением в окружающей дейтсвительности
	\end{itemize}
\end{enumerate}

По субъекту деятельности:
\begin{enumerate}
	\item Индивидуальная 
	\item Коллективная 
\end{enumerate}

По характеру деятельности:
\begin{enumerate}
	\item Репродуктивная -- создание чего-то по шаблону
	\item Творческая -- описано выше
\end{enumerate}

В соответствии с правовыми нормами:
\begin{enumerate}
	\item Законная
	\item Незаконная
\end{enumerate}

В соответствии с моральными нормами:
\begin{enumerate}
	\item Моральная
	\item Аморальная
\end{enumerate}

В соотсветствии с общественным прогрессом:
\begin{enumerate}
	\item Прогрессиваня
	\item Регрессивная (реакционная)
\end{enumerate}

В соотвествии с сферой деятельности и или сферой жмзни общества:
\begin{enumerate}
	\item Экономическая
	\item Политическая
	\item Социальная
	\item Духовная
\end{enumerate}

По особенностям проявления экономической активности:
\begin{enumerate}
	\item Внешняя (мышечная)
	\item Внутренняя -- внутри сознания (умственная)
\end{enumerate}

\begin{Huge}
Проверочная	
\end{Huge}\\
Схема Сочинение
\begin{itemize}
	\item Перефраз
	\item Точка зрение ($+, -, \pm$)(согласен/несогласен/частично согласен)
	\item Аргументация из пройденного курса
	\item Аргуменнтация из истории, искусства и личного опыта
	\item Заключение
\end{itemize}
\end{document}