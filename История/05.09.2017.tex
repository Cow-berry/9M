gjp	 \documentclass[12pt,a4paper]{article}
 \usepackage[utf8]{inputenc}
 \usepackage[russian]{babel}
 \usepackage[OT1]{fontenc}
 \usepackage{amsmath}
 \usepackage{amsfonts}
 \usepackage{amssymb}
 \usepackage{graphicx}
 \usepackage[left=2cm,right=2cm,top=2cm,bottom=2cm]{geometry}
 \author{Анатолий Коченюк}
 \title{Конспект по истории}
 \begin{document}
 \section{9 класс}
\subsection{Зарождение}
Восточно-Европейская равнина.\\
Важные Реки : Дон, Днепр, Волга.\\
Из кого стало создаватся Русское государство : Восточные славяне(древляне, поляне, родимичи, кривичи, словене, вятичи.\\
В какой момент начинают формироватся прообразы государства(племенные союзы) -- VI-VII века\\
Дата зарождения государственности :\\
862 год -- приглашение Рюрика.\\
\begin{enumerate}
\item Самая устоявшеяся "Норманская гипотеза", что Рюрик -- варяг(выходец из Скандинавии)
\item Рюрик мог быть новгородской элитой
\item Рюрик мог оказаться дальним родствеником <...>
\end{enumerate}
879 -- конец правления Рюрика\\
879-- начало правления Олега (из окружения Рюрика, возможно родственик). Он правил до того, как вырос Игорь.\\
Олег -- первый Киевский князь(распространил своё правление на Киев и прославился походами на Константинопль).\\
Прозвище Олега -- Вещий Олег.\\
912 год -- конец правления Олега.\\
912 год -- начало правления Игоря\\
Игорь хотел собрать с древлян дань 2 раза , за что его убили привязав между двумя деревьями и отпустив их, разогнув.\\
945 -- начало правления Ольги.\\
Ольга -- приняла христианство, сделала Римкого Императориа своим крёстным отцом. Также она изменила систему сбора дани(погосты).\\
964 -- Святослав - 972\\
Прославился своими военными походами (Хазаре : уничтожение, Византия)\\
В Византии пришлось идти на переговоры с Византийцами, оставить всё, что он захватил. Но на одном из привалов он был убит степным народом - печенегами.\\
После смерти Ярослава:\\
972-980 -- междуусобицы сыновей Святослава( Ярополка, Олег и Владимира)\\
980 -- Владимир (Великий, Красное Солнышко, Святой)- 1015\\
Был самого низкого происхождения(сыном ключницы), но в итоге вокнязился, устранив своих братьев.\\
Нужна какая-то объединяющая вещь -- нужна религия. Он подвёл к себе идолов из разных частей Русской земли, но людям это не понравилось.\\
Легендарная версия(Отвергал Иудаизм, ...). И он понял, что лучше всех них(а также лучшим союзником) будет Византия\\
988 год - год начала крещения Руси. Не проходило за один день. Некоторые сохраняли Свою религию вплоть до XV-XVI века. Проходило не всегда добровально.\\
1019-1019 -- новая междуусобица : Святополк, Ярослав, Борис и Глеб
Юрак с Византийской царевной Анной : Борис и Глеб.\\
Святополк -- Святополк Окаянный. Погиб в 1019 году \\
Юорис и Глеб -- первые святые(родились крестьянами)\\
Ярослав -- Ярослав Мудрый. Почему :
\begin{itemize}
	\item Создание свода законов "Русское правда
	\item Дипломатия -- около 9 своих детей он женил на принцах/принцессах соседних стран
	\item Строительство церквей и монастырей(центры культуры : учат писать, знакомят с  трудами философов)((3 Софии -- )).
\end{itemize}
После смерти Ярослава власть перешла к его сыновьям, которые довольно долго правили во взаимипонимании. Триумвират Ярославичей-- совместное мирное сосуществование.\\
После 1070 годов Триумвират развалился -- начало междуусобиц\\
Итогом междуусобиц конца XI века стал Любечский съезд 1097 года, итогом которогог стал принцип : "Каждый да держит вотчину свою".\\
Правивший в 1113--1125 годах Владимир Мономах и его приемник(сын)\\ Мстислав 1125--1132 стали последними правителеми Древней Руси, как единого государтсва.\\
Многие центры Древней Руси начинают обособляться\\
Земли :
\begin{itemize}
 \item Киевское (тот, кто  там правит -- маиый главный князь)
 \item Новгородское((там было вече(прообраз парламента), в отличие от остальных княжеств приглашали на свой престол других князей, был очень высокий уровень образованности -- берестяные грамоты умели писать самые обычные жители)
 \item Владимиро-Суздальское(за XII-ый век ставшее наиболее сильным княжеством, большое количество потомков мужского рода)
 \item Полоцкое(современная Беларусь, именно Полоцокое княжество обособилось раньше всех, правили потомки Владимира от одного из братьев Ярослава)
 \item Галицко-Волынское (в нём был КОРОЛЬ, до них не дошли татары)
\end{itemize}
\subsection{Нашествие монголо-татар}
В начале XIII-го века начинает формироваться монгольское государство\\
1207 год -- лидер этого государства Тимуджин становится Чингизханом \\
к 1220 годам -- весь Северный Китай и центральная Азия -- завоёваны.\\
1223 год -- первое столкновение русских князей с монголами на реке Калка\\
в 1223 монголы дальше не пошли и вернулись в свои улусы
1227 -- смерть Чингисхана - Батый\\
На западе было монгольское Последнее Море и они хотели дойти до него и верилди, что если они дойдут, то наступит благодать.\\
1236 -- Волжская Булгария больше на картах не появляется\\
1238 -- Северо-Восточная Русь\\
1240 -- Южная Русь\\
После разорения южной Руси монголы проходят через Польшу и Чехию и доходят до Адрианского моря. Из-за того, что стали доходить вести о избирании нового хана - Батый сказал, что это послднее море, и они повернули обратно\\

\subsection{Московское княжетсво в XII-XIV веках}
Из московского княжетсва в итоге выросло то государство, которое сквозь все преобразования, изменения строев и т.п. выросло  в государство, сохранившееся до наших дней.\\
Первым князем, правившим 40 лет  является Даниил Александрович Московский.\\
Первый князь, сделавший Москву своим уделом и она стала столицей его княжетсва.\\
Даниил будучи сыном такого значимого человека, как Александр, тоже 
1293 -- "Дноденева Рать"\\
Территориальные приращения Москвы : Переяславль -Залесский(чуть севернее)\\
При Данииле Московском был построен первый Московский Кремль с большими деревянными башнями.\\
Ныняшний кремль -- самая большая среденвековая крепость Европы.\\
Иван Даниилович (сын Даниила) / Иван I Калита(кошелёк): 1325 -- 1340 .\\
\begin{itemize}
	\item прозвище, указывающее на щедрость
	\item был человеком "умеющим считать деньги"
	\item Начал покупать уделы северо-востока Руси
	\item был кошельком для Монгольского хана
\end{itemize}
 При нём :
 \begin{enumerate}
	\item престол стал передаваться от отца к сыну(не по лестничному праву)
	\item Покупка уделов	
	\item 1327 -- организация переезда в Москву Митрополита всея Руси	Алексий
	\item Калита был женат на дочери Монгольского хана Узбека(имя) $\Rightarrow$ доверия к нему $\Rightarrow$ назначение его сборщиком дани с северо-востоной Руси
\end{enumerate}
Коломна присоеденилась к Московскому княжеству при Калите\\
После Ивана Калиты:\\
1340 -- 1353 Симеон Гордый\\
1353 -- 1359 Иван II Красный\\
1359 -- 1389 Дмитрий Иванович Донской\\
1350-е -- 1380-е 	"Великая Замятня" в Золотой Орде : сменилось на троне 25 ханов\\
при Дмитрии был построен белокаменный Кремль. Возможность такого капитального строительства подразумевала, что княжество может себе позволить такое построить.\\
Московское княжетсво += Кострома, Калуга, Владимир\\ 
1377 -- битва на реке Пьянь\\
1378 -- сам Дмитрий выступает против Орды, отказывается отдавать дань	\\
Темник Мамай(тьма = 10000) был крупнейшим из Монгольских военачальников. Дмитрий Донской выступал против него, аргументирую это тем, что они не Чингизит(потомок Чингизхана). Он понимает,что его могут атаковать и собирает армию.\\
К 	1380  он формирует коалицию во главе с князем Дмитрием Ивановичем Московским с целью отражения нападения монголов.\\
Даже не входящие в Московское княжетсво Тверское и Новгородское княжетсва присылают некоторое количество солдат.\\
Войско было собрано в Коломне и по разным оценкам насчитывало от 35 000 до 50 000 -> р. Дон -> поддержка Литовцев
\subsection{Великое княжество Литовское}
Литва -- смесь славяно - балтийская. Она всячески придавливается со стороны Тевтонских рыцарей.\\
Сер. XII в. Миндовг(Мяндоуг, Mindaugas) собирает литовские племена и в 1252 принимает крещение, как КОТолик.\\
Князья Литвы обращают свой взгляд на земли западной и южной Руси.  \\
1314 - 1341 -- Гедимин(Hiedziminas) за время его правления территория увеличивается в 3 раза за счёт Полоцкого, Минского и Турово-Пинского княжества.\\
1341-- 1377 Приемник Гедемина -- Ольгерд(Альгерд, Aligirdas)
\begin{itemize}
	\item Киев
	\item Чернигов
	\item Смоленск
\end{itemize}
Столица -- Вильнус\\
Трижды ходит на Москву.\\
Литовцы -- язычники\\
80\% населения -- правослвные русисичи\\
Главный язык на территории Литовского княжества -- западно-Русский(похож на белорусский)\\
XIV XV века Западная Русь обособляется от востока\\
Имела претензии к Древней Руси\\
Витовт(Витоут , Vitautas) и Ягайло(Ягелло)\\
1380 6 сентяря -- Куликовская битва(~ 60 000 монголов)\\
1385 -- династическая уния между ВКЛ и Польшей	\\
Челубей и Пересвет убили друг друга. Монголы попытались пробить центр Русского войска\\
1380 -- войско Донского победило.\\
Мамаю прилось бежать и его не принимали в Орде\\
Следущий правитель в Орде -- Тохтамыш\\
1382 -- поход на Моску и её сжигание -> принуждение платить дань
{\Huge Проверочная : Даниил Московский -- конец конспекта + посмотреть Княжество Литовского}
\\
Василий I (1371 (рождение) -- 1389 (престол) -- 1425 (смерть))\\
4 года в раннем возрасте Василий провёл в плену у Хана\\
1386 год -- Василий I покидает хана Тохтамыша\\
1391 год -- жениться на Софье Витовне (дочь великого князя Литовского) \\
1389 -- восход на трон\\
1392-1393 -- Ниж. Новгород и Муром присоеденяются, училение Москвы\\
1395 -- По золотой Орде проходит поход Тамерлана (Тимура (с тюркского : Железо)). Улус Чагутаев -- осколок З.О.\\
в конце XIV  совершает походы в Индию и Персию. Считается, что он на некоторое время спас Константинопль.\\
1400 -- битва при Анкаре.\\
1395 -- Поражение Тохтамыша $\Rightarrow$ Елец\\
Конфликтов с Литвой долгое время не было, про причине родства Витовта и Василия.\\
1403 -- 1404 -- Начало противоречия : Смоленское княжество.Витовт, путём различных политических игр смог присоеденить Смоленское княжество к Литве.\\
В конце XIV века Литва приняла католичество.\\
Привелегии за смену религии.
Многие дворянские рода Литвы начинают переходить под власть Московского князя.\\
1407 -- начало боевых действий.
ВКЛ спорило с МК спорило за земли. Хан Едигей(Ставленник Тимура З.О.) решило воспользоваться ситуацией.
Пока войска МК и ВКЛ стояли на реке Угре, Хан Едигей атаковал ВКЛ с южных границ.\\
1408 -- поход Едигея на Москву\\
Едигей не достиг всего, чего хотел, но снял с Москву  "небольшой" выкуп.\\
Мир с Литвой. Литва сохранила своё влияние на Московское княжество. Москва и Литва пытаются \textit{коммуницировать}\\
Литва пероре=иентировала все свои силы на запад.\\
1409 -- 1411 -- Великая война с Тевтонским орденом. Были впервые применены массого артилерия и огнестрельное оружие.\\
Главные противники Тевтонского рордена -- Польша и Литва (совместная война)\\
1410 -- Грюнвальдская битва (~25 т. Тевтонских рыцарей, ~30-35 т. пол.-лит.)\\
Фон Юнгинген vs Ягайло  + Витовт\\
Понижение роли Тевтонского ордена -- Повышение роли Польши и Литвы.\\
Конец XIV -- начало XV века -- время распада Золотой Орды $\rightarrow$ Казанское, Крымское, Сибирское Ханство, Нагайская Орда, Большая Орда. Распад Орды привёл к изменению отношений между Московским княжествам и частями
\subsubsection*{Василий II}
Василий II -- 1415 --  самый младший сын Василия I и Софьи Витовны, но старшие сыновья не дожили.\\
1425 -- правление\\
В первые годы правления Василий находился под опекой своей матери.\\
1430 -- смерть Витовта\\
 После смерти Витовта родственники Василия постепенно перестают подчиняться его власти. \\э
 Дмитрий Донской
\begin{tabular}{c||c}
Василийй I & Юрий Дмитриевич\\
Василий II & Василий Косой + Дмитрий Шеляка\\
\end{tabular}
1431 Юрий отказывается пожчиняться и посылает посольство в Орды с просьюой получить Ярлык на княжение в Москве.
1433 -- Свадьба Василия II. На ней присутсвуют братья Василия. В разгар цремонии Софья Витовна "случайно" замечает пояс,  который должен был быть подарен Дмитрию донскому на его свадьбу. Юрьевичи были вынуждены покинуть свадьбу и Москву... Василий уже утвердился в Москве, казнили одного из сподвижников Юрьевичей в Москве.Пришли Юрьевичи уже с войском.
1433-1434 -- дважды воцаряется на троне в москве Юрий Дмитриевич. Умирает на престоле и завещает престол Василию II. Василий Косой против этого. В ходе пртивостояния Василию Косому выкололи глаз(а может и не один). После этого Василий  II в очередной раз воцарятся на престол.\\
1445 -- 1447 -- правитель Московского Княжества -- Дмитрий Шемяка. 1445 -- 1446 -- Ваилий II получил помощь от Хасимонского ханства. Когда Дмитрий снова начал компанию по восстановлению своей власти . Обвинение : "ты привёл на Русь татар, ты враг для Руси". И выкалывает Василию II (Тёмный) глаза. Василий II -- снова становиться князем 1447 -- 1462.
Д/З :
\begin{Huge}
1448 -- Епископ Иона \\
1439 -- Флорентийская Уния \\
1453 -- падения Канстантинопля
\end{Huge}
\begin{center}
	Епископ Иона 
\end{center}
Митрополит Киевский и всея Руси\\
С 1431 года — епископ Рязанский и Муромский.\\
15 декабря 1448 года на Соборе русских епископов в Москве был избран митрополитом Киевским и всея Руси. Его кандидатура не была согласована с Патриархом в Константинополе; однако предполагалось, что он получит благословение в Царьграде post factum. Но епископу Ионе не удалось пойти в Царьград за посвящением.
\begin{center}
	Флорентийская Уния
\end{center}
Флорентийская уния 1439 г. — соглашение между представителями католической и православной церквей, заключенное на Ферраро-флорентийском соборе 1438—1445 гг., на условиях признания верховенства папы над всеми христианами в обмен на обещанную помощь в отражении османской агрессии против Византии. Уния не вошла в силу ни в Византии, ни в России, хотя Московский митрополит Исидор и поставил под ней свою подпись.
\begin{center}
	падение Константинополя
\end{center}
адение Константинополя в 1453 году — захват столицы Византийской империи Константинополя турками-османами под предводительством султана Мехмеда II во вторник, 29 мая 1453 года. Это означало уничтожение Восточной Римской империи, последний византийский император Константин XI Драгаш пал в битве. Победа обеспечила туркам господство в бассейне Восточного Средиземноморья. Город оставался столицей Османской империи вплоть до её распада в 1922 году.

Хотя Константин XI пытался привлечь на свою сторону всех христиан, включая католиков, и всячески защищал Флорентийскую унию, считая её необходимой для спасения Византийской империи перед лицом турецкой угрозы, он не смог преодолеть оппозиционных настроений внутри страны, особенно среди среднего и младшего клира[2]. Дело дошло до того, что один из самых высокопоставленных военачальников, командующий византийским флотом мегадука Лука Нотарас открыто заявлял, что, по его мнению, «лучше увидеть в городе царствующей турецкую чалму, чем латинскую тиару». Таким образом, уния не привела к оказанию Византии сколь-нибудь значительной помощи со стороны католиков, но зато весьма ослабила ее внутренне политическое единство, вызвав в стране глубокий раскол на сторонников и противников унии.

При этом империей Византию середины ХV века можно было считать лишь условно: на момент падения города власть императора ограничивалась лишь крепостными стенами Константинополя, а его население не превышало 50 тыс. человек.

{\Huge д/з московское кн-во XII-XVI вв.\\ Писать конспекты\\Сочинение : противопостояние Москвы и Новгорода. Что нужно написать : 
\begin{itemize}
\item основные различие (государственного устройства и политической жизни) 
\item причины побды Москвы расширять и пояснять тезисы
\item оценка победы Москвы (влияние победы на дальнейшее развитие России (позитивное/негативное))
\end{itemize}
4 * A4 + титульный лист
12 шрифт}

Иван III (1400--1462--1505)

Сын Василия II, детство проходило на фоне междуусобиц. 

После ослепления отца Иван III является практическим соправителем.

1462 -- восход на престол, его отец в конце жизни болел туберкулёзом. Иван получил московское княжение(16 главных городом Московского княжества), а его младшие братья
 получили вотчины
\begin{enumerate}
\item Собирание земель, был заложен Иваном Калитой
процесс присоеденения земель древней Руси (Северо-Восточных) с XII по  $XVI^*$
1471, 1474 -- Иван присоединяет к Москве Ярославль и Ростов. Пока женой Ивана является Тверская княжна.

Новгород попадает под влияние Москвы 

1471 Новгород восстаёт против влияния Москвы.  Иван попытался утвердить своего родственника. новгородцы начинают договариваться с литовцами

Марфа Борецкая (посадница) -- лидер антимосковской партии. Иван отправил на Новгород войска.

1461-- битва на реке Шелон. Московская армия нанесла поражение. Для Новгорода поражение в этой битве означало потерю самостоятельности но формально как государство Новгород продолжал существовать.

Вторым этапом этого противостояния 
1477-- ко двору Ивана III прибыли жаловщики из Новгорода, которые назвали его не господин, а государь т.е. они поставили его выше Новгорода, а не рядом  с ним. Иван потребовал подчинения у Новгорода и в 1478 послал большое войско, после чего отправил Марфу в ссылку. Из Новгорода был унесён вечевой колокол. После этого Новгород попал под прямую зависимость от Москвы

1471 -- жена из Твери умерла. Тверь потеряла дополнительную защиту. 

1479 -- Тверь была проведена к присяге 

1485 -- Тверь вошла в Московское княжество 

Внутренняя политика Ивана. Основной целью была централизация власти , урезание прав аристократии, введение наместников,  людей подвластных Ивану  ,начали происходить массовые переселения (самые крупные аристократические семьи переезжали в тыл Москвы где сливались с обычным населением 

1497 -- Судебник -- унификация всех законов русских земель. введение Юрьева дня(26.11). В этот момент крестьянин может выплатить родила и перейти к другому помещику.  С этого момента начинается закрепощение. За беглыми крестьянами уже существовал сыск 
\item Внешняя политика
Орда
с 1470 - х -- начало отстранения. находит союзников в Крымском ханстве 
большая Орда выступает против Москвы
1480 войско князя встречает войско Орды на реке Угра и происходит сражение. У князя была артиллерия. Войску Ордынцев пришлось отступить. Конец татаро-монгольского Ига.

другими векторами внешней политики: войны с великим княжеством литовским,  по итогам у Литвы были отобраны значительные территории (современная Смоленская Калужская Брянская области). 

войны со Швецией,  их итогом стало сохранение статус-кво (сохранение итогового полоения). в нашей области есть памятник этому противостояния. Иван в ходе внешней политики активно участвует в назначении хана в Казани.  на престоле оказываются люди появившиеся там не без участия московского князя.  

главным итогом правления Ивана стало формирование единого государства, снятие зависимости от орды и становление самодержавной власти, за все его поступки Иван III вошел в историю ,как Иван Великий
\end{enumerate}

%к/р 
%1) Начало возвышения Московского княжества. Иван I
%2) Княжества-соперники Москвы.
%3) Выход Москвы из Ордынской зависимости. Процесс и события
%4) Великое княжество Литовское. Образрвание, возвышение, отношения с другими княжествами в том числе с московским, остальные персорналии

\begin{Huge}
Д/з Татаро-Монгольское иго, про и контра (за и против)

ввод в курс дела .  В исторической науке есть два противоположных взгляда на иго:

иго разорило северо-восточную русь остановило культуру и т.п и очень жал что случилось

разорение имело меесто быть, но под покровительством ига Русь могла сохранить такую свою бытность и вообще сохранилась


что нужно написать :
\begin{enumerate}
	\item	Точки зрения
	\item	Аргументы
	\item	Ваше мнение
\end{enumerate}

экспресс тест по теме по тому что мы писали + ПРИНЕСТИ КОНСПЕКТ 
\end{Huge}

Наверху пирамиды власти находился\textbf{ Великий Князь} (государь)\\
под ним находится \textbf{Боярская дума} (около 20 человек)\\
под ней находились \textbf{Наместники}\\
ниже наместников, в уездах (бывших наделов) располагались воласти и станы. Там располагались \textbf{волостели}.\\

В занятии государственных должностей было 2 принципа:
\begin{itemize}
	\item Кормление -- он кормится с того, чем занимается. К примеру кто-то собрал дани на 10 сундуков и один забрал сразу себе. Провацировало взяточничество.
	\item Местничество -- принцип занятия государственных должностей в соответсвии с о знатностью рода.
\end{itemize}

Иван III создаёт систему поместий. Поместье -- земля, кототрое получал человек за службу Великому князю. Человек владеющий поместьем -- поместник.\\
Из помещиков был составлен ударный костяк -- Поместное войско.\\
Конно, Людно и Оружно -- в случае войны он должен прийти вооружённым, со своим оружием.\\
\subsection{Ереси (Религиозная борьба) в Московском княжетсве в XV веке}
Новгород был городом свободным и открытым для иностранцев. Многие проповедчики жили и чуствовали себя там хорошо. 

В конце XV века в Новгороде фофрмируется ряд рилигиозных течений.

Первой из них становится течение стрегольников (особенности внешности)
Выступали с антикриликалами. Они выступают против всех религий.
Стали отрицать некоторые христиански енормы. Они были разогнаны.

Второй -- Ересь Жидовствующих $\rightarrow$ Ересь Новгородствующих. Было много элементов Иудаизма. Отрицали, что Иисус -- бог и отрицали святую троицу.

Она перекочовывает из Новгорода в Москву

 Спор Нестяжателей (заточен на то, чтобы всего стало юльше -- религиозное движение, выступавшее за сокращение церковной собственности. Не только оклады и кадила золотые, но ещё и земли) и Иосяфлян ("собственность церкви нужна для распространения Православия")
 Руководитель Иосефлян -- Иосиф Волоцкий.
 
 Поначалу Иосефляне проигрывали Нестяжатлям. В итоге победа таки осталась за Иосефлян, т.к. они изменили свою похицию за Великого князя.
 
 \subsection{Идеология Московского государства}
 Во времена Ивана III формируются предпосылки концепции "Москва -- третий Рим" (сформировалась в 1530 году). 
 
 Каки факты послужили формированию:
 \begin{itemize}
 	\item Фактор формирования -- Московская Русь -- православная страна. Это даёт уникальный момент того, что оно единственно таковое.
 	\item 1453 -- падение константинопля.
 	\item 1439 -- Византия подписала флорентийскую унию с папой Римским. В этой унии Византия признавала главенство католической церкви. Униатское течение -- стремление объеденить Православную и Католиеску церковь полд началом Папы Римкого.
 	\item 1448 -- Митрополит Иона -- избран собором церковной руси, а не с помощью Византии.
 	\item 1472 -- Иван III женился во второй раз и его избранницей стала Зоя (Софья) Палеолог -- племнница последнего визвнтийского императора\\Москва -- второй Костантинопль\\
 	Рим I -- в нём уже другая религя (впал в Ересь)\\
 	Рим II = Константинопль -- признал власть иноверцев\\ 
 	Рим III = Москва -- страна не поаисывающяя никаких уний, имеющяя истинную веру ...\\
 	"Третий Рим стоит, а четвёртому не бывать"
 \end{itemize}
 С правления Ивана III закрепляется позиция, которая выше всех остальных. 
 Самодержец -- калька из Византийского обряда. Его называли Автократом.
 При дворе Ивана III начинается Визнатийской церемониал (бояре целуют руку князю).
 
 Появляется символ России -- двуглавый орёл. Считается, что Русь воспринимает символ Византии. Но он используется у некоторых степных народов\\
 
 При нём формируется Успенский собор. Архитектором его был Аристотель Фиораванти. В этом соборе короновались все правители.
 
 Кремль -- создан Солари и Руффо. 
 
 Грановитая алата -- место для различныз заседаний.
 
 Архангельский собор -- 1506 год 
 
 Престолонаследие.
 От первого брака Ивана появился Иван Молодой. Многие исследователи считают, что о являетс я прототипом Ивана Царевича. У него был Дмитрий Внук -- сын Ивана Молодого.
 
 от Софьи был Василий III. Был раскрыт заговор нацеоенный на убийство Дмитрия
 1499 -- коронация своего внука Дмитрия, как юудущего правителя.
 
 \subsection{Василий III}
 годы правления : 1505 -- 1533\\
 Старший сын Ивана от Софьи Палеонтолог. Наполовину византиец\\
 1490 -- Иван Молодой Умирает. Кокурения за наследование. Царевич Дмитрий Внук\\
1499 -- был коронован Дмитрий\\
1502 -- был коронован Василий. запер в тюрьму Дмитрия, который позже был похоронен в Архангельский собор.\\
1505 -- смерть Ивана III. Воцарение Василия III\\
при Василии III Московское княжетсво напоминает Европейскую часть России\\\
1510 -- присоединение Пскова. Псковский вечевой колокол был снять\\
1517 -- Василий III подавляет независимость Рязанского княжества. Присоединение Рязани.\\
300-400 важных семей были разбросаны по Руси.\\
Василий всячески ограничивает любые права боярской аристократии по схеме своего отца\\
\paragraph{Внешняя Политика}
Война с Литвой.\\
1512 -- начало войны \\
1514 -- захват Смоленска. Планы дальнейшего наступления на Литву.\\
1514 -- битва под Оршей. Константин Острожский('белорус') ведёт противника а бой. \\
1522 -- конец войны. Главный итог -- завкрепление границ.\\

\begin{enumerate}
	\item Сигизмунд Губерштейн -- оставил очень большие и интересный для исследователей записки с описаниями Московского княжества.
	\item Посол СРИ (священной Римской Империи) назвал Ивана III императором. Через 200 лет Пётр I называет себя Императором, т.к. нашего деда Римская Империя назвала Императором. Царь -- калька от слова цезарь.
\end{enumerate}
\paragraph{Семья}
При восшествии на престол первой женой Василия III становится Саломония Сабурова. Впервые на Руси применяется традиция смотра Невест (Византийская). \\
Выбиралось около 1500 дворянок, не очень богатых. Саломония не княжеского рода и не инностранка. Пробыла женой Василия 20 лет. Саломония никого не родила и Василий отправил её в монастырь.\\
2-ой женой Василия III стала Елена Глинская. Родиллась и выросла в Литве, но перешла на стороны Московского княжества. \\
Сыновья -- Иван IV Грозный и Юрий (Георгий)\\\\
\emph{Столетняя Война 1337 -- 1453\\
Причины
\begin{enumerate}
	\item Территории(Англия -- западная Франция, Франция -- Гасконь)
	\item Династические претензии. Жак Де Мале горя проклял Филипа IV . Филлип Красивый (IV) успел родить 4 ребёнка: Луи X (умер в 1316), Филлип V (умер в 1322), Карл (умер в 1328), Изабелла (стала Английской Королевой, женой Английского короля Эдуарда II. Они родили Эдуарда III)\\ Королём Француы решают назначить \\
\end{enumerate}
1337 -- начало войны\\
1340 -- Анличане рядом с городом Слейс наносят морское поражение французам и открывают себе дорогу на континент.\\
1346 -- битва при Креси -- первое крупное поражение французов в столентней войне. В большей степени из-за недисциплинированности. \\
1347 -- 1352 --- великая чума. на енбольшое время останавливает войну.
1356 -- возобновление военных действий. Сражение при Пуатье. Англичане высадились на западе. Новый Французкий король противостоит знаменитому Английскому военасальнику Эдуарду Чёрному принцу. Французы повторяют те же ошибки. Иоанн II французкий король попадает в плен. Выставляют большую сумму выкупа. Объявляется дань на это. 
1358 --- Жакерия -- крупный народный бунт.\\
1360 -- перемирие с Англичанами. Огромная рповинция Аквитания переходит под правление английского короля}

\textbf{{\Huge Будет проверочная по московскому княжеству}}

После заключения перемерия король возвращается на Родину и начинает подготовку к отбиванию территории.

1360 -- французкое войско под руководством военачальника Дугеклена начинает войну против Англии. Войску Дюгеклена удалось отвоевать почти все английские владения. А для Англии ситуация усугубилась тем, что Эдуард Чёрный принц умирает, а затем умирает и Эдуард III -- его отец

1380 -- против Англии начинает выступать Шотландия

1381 -- в Анлии начинается восстание Уота Тайлера

Итогом всех этих неурядиц для Англии стало заключение перемирия. Но особенных выгод Франция не получила. 

1390 -- начало правления Карла VI. Имел психические расстройства. За престол боролись бургунды(восток и северо-восток) и арманьяки (центр). В своих рапрях не боялись использовать Английскую помощь. 

Междуусобицы ослабили Францию, что привело к тому, что англичане начали новую кампанию против Франции. На сторону англичан встала герцогство Бургундское (Бургундия). 

Первым сражением этой войны стало сражение при Азенкуре. Анличане пользовались прототипом регулярной армии. Франция же использовала аристократическое войско (рыцарей). Но за день до сражения прошёл дождь и Французкая конница начала там вязнуть. Англия же, тне долго думав, начала обстреливать их.
Итогом поражения при Азенкуре стало отступление Французов с севера Франции и даже оставили такие города, как париж и Реймс.

1420 -- договор в Труа. Итогом этого договора стало закрепление севера Франции за Англией

По этому договру Французкий король должен был передать свой престол сыну Английского короля тех времён. Французкий наследник Дофин (его герб -- дельфин)

1423 -- сын Английского короля был коронован. Королями Франции Английские короли перестали быть лишь тогда , когда Франци перестала быть королевством.  сэтого момента Франция была в крайне тяжёлом положении, а её площадь сократилась в 2 раза по сравнению с предвоенными временами. Дофин Карл укрывался на подвластной ему территории и не имел шансов на престол.

Финальный этап войны был связан с именем Жанны Д'Арк (1421 -- 1431). 

1429 -- Жанна прибывает в дворец Дойина, по легенде узнаёт среди толпы

Дофин режил передать под её руководство отряд.

1429 -- отряд Жанны вступает в английскую крепость Туррель и втечение следущего месяца она отакует ряд английских укреплений. Идёт на Реймс. После удачного похода на Реймс уже с Дофином Карлом. После коронации Карл утрачивает интерес к Жанне и больших отрядов не выдаёт.

1431 -- неудачная попытка Штурма -- Жанна попадает в плен к бургундцам. Они предлагают её выкупить, но Карл отказывается. И они продают её Англичанам.

1431 -- сжигание на костре Жанны Англичанами. Французы обиделись.

1439 -- почти вся территория Франции освобождена, в частности Париж. Французы переключаютя на западные рубежи и атакуют Гасконь. И после взятия Бордо изгоняют англичан и оттуда. 

1453 -- завершение войны. За англичанами по итогу войны остался лишь город Кале под севером Франции (сейчас уже французкий).
\\\\\\
Многими исследователями завершение столетней войны считается завершение эпохи средневековья. Феодальные армии стали вытеснятся регулярными. Формирование Национального самосознания и Национального государства.

\subsection{Завершение формирования государств}
\subsubsection{Франция}

Бургундия осталась достаточно независимой. Борьбу с Бургундией осуществлял Французкий король Людовик XI (1461 -- 1483) vs Карл Смелый. После гибели Карла Людовик устанавлиапет полный контроль над Францией

\subsubsection{Англия}
После поражения Англия переживала кризис, связанный с очень большиими потерями. В стране остаётся очень много людей, лишившихся основной своей деятельности.

\subsection{Реконкиста во Франции}
Рекнкиста (от исп. отвоевание) процесс изгнания арабских завоевателей с Пиринейского полуострова. Арабский халифат закрпился на территории Испании. 
Реконкиста длилась более 700 лет (711 -- 1492).

Её завершение ббыло связано с именами Фердинандом Арагонским и Изабелла Кастильская. Спустя какое-то время они даже стали мужем и женой.

1492 -- объединёнными усилиями нанесли поражение последним опорным пунктом арабов -- Гранадскому эмираты. Брак Изабеллы и Фердинанды в итоге положит начало объединённому Испанскому государстсву. С правлением Изабеллы и Фердинанда связывается активная леятельсность инквизиции. 

Польша не изгоняла Евреев. В итоге Евреёскаячасть населения больше трети... Фердинанд и Изабелла стали спонсорами исследовательских экспедиций конца XV века

\subsection{Война Алой и Белой розы}


война Алой и Белой Розы
1455 -- выступление Ричарда Йорк -- родственнник короля
Ричард начал выстуаления против Генриха
Причины войны

Причинами войны послужили тяжелое экономическое положение Англии (кризис крупного вотчинного хозяйства и падение его доходности) поражение Англии в Столетней войне (1453), лишившее феодалов возможности грабить земли Франции; подавление восстания Джека Кэда в 1451 (см. Кэда Джека восстание) и вместе с ним — сил, выступавших против феодальной анархии. Ланкастеры опирались главным образом на баронов отсталого севера, Уэльса и Ирландии, Йорки — на феодалов экономически более развитого юго-востока Англии. Среднее дворянство, купцы и зажиточные горожане, заинтересованные в свободном развитии торговли и ремесел, ликвидации феодальной анархии и установлении твердой власти, поддерживали Йорков.

При слабоумном короле Генрихе VI Ланкастере (1422-61) страной правила клика нескольких крупных феодалов, что возбудило недовольство в остальных слоях населения. Пользуясь этим недовольством, Ричард, герцог Йоркский, собрал вокруг себя своих вассалов и пошел с ними на Лондон. В битве при Сент-Олбансе 22 мая 1455 он разбил сторонников Алой розы. Отстраненный вскоре от власти, он вновь поднял мятеж и заявил о своих притязаниях на английский престол. С войском своих приверженцев он одержал победы над противником при Блор-Хите (23 сентября 1459) и Норт-хемптоне (10 июля 1460); во время последней он захватил в плен короля, после чего заставил верхнюю палату признать себя протектором государства и наследником престола. Но королева Маргарита, жена Генриха VI, со своими приверженцами неожиданно напала на него при Уэйкфилде (30 декабря 1460). Ричард был разбит наголову и пал в сражении. Враги отрубили ему голову и выставили ее на стене Йорка в бумажной короне. Сын его Эдуард при поддержке графа Уорика разбил сторонников Ланкастерской династии при Мортимерс-Кроссе (2 февраля 1461) и Тоутоне (29 марта 1461). Генрих VI был низложен; он и Маргарита бежали в Шотландию. Победитель стал королем Эдуардом IV.

\textbf{Эдуард IV}

Однако война продолжалась. В 1464 Эдуард IV нанес поражение сторонникам Ланкастеров на севере Англии. Генрих VI был взят в плен и заключен в Тауэр. Стремление Эдуарда IV усилить свою власть и ограничить свободы феодальной знати привело к восстанию его прежних сторонников во главе с Уориком (1470). Эдуард бежал из Англии, Генрих VI в октябре 1470 был восстановлен на престоле. В 1471 Эдуард IV при Барнете (14 апреля) и Тьюксбери (4 мая) разбил войско Уорика и войско жены Генриха VI Маргариты, высадившееся в Англии при поддержке французского короля Людовика XI. Уорик был убит, Генрих VI в апреле 1471 вновь низложен и умер (предположительно убит) в Тауэре 21 мая 1471.

\textbf{Завершение войны}

После победы, дабы укрепить свою власть, Эдуард IV начал жестокие расправы как с представителями Ланкастерской династии, так и с мятежными Йорками и их сторонниками. После смерти Эдуарда IV 9 апреля 1483 престол перешел к его малолетнему сыну Эдуарду V, но власть захватил младший брат Эдуарда IV, будущий король Ричард III, который сначала объявил себя протектором малолетнего короля, а потом низложил его и приказал задушить в Тауэре вместе с его младшим братом Ричардом (август (?) 1483). Попытки Ричарда III укрепить свою власть вызвали восстания феодальных магнатов. Казни и конфискации имущества восстановили против него сторонников обеих группировок. Обе династии, Ланкастеров и Йорков, объединились вокруг Генриха Тюдора, дальнего родственника Ланкастеров, который жил во Франции при дворе короля Карла VIII. 7 или 8 августа 1485 Генрих высадился в Милфорд-Хейвене, беспрепятственно прошел через Уэльс и соединился со своими сторонниками. От их объединенного войска Ричард III потерпел поражение в сражении при Босворте 22 августа 1485; сам он был убит. Королем стал Генрих VII, основатель династии Тюдоров. Женившись на дочери Эдуарда IV Елизавете — наследнице Йорков, он объединил в своем гербе алую и белую розы.

\textbf{Итоги войны}

Война Алой и Белой роз явилась последним разгулом феодальной анархии перед установлением абсолютизма в Англии. Она велась со страшным ожесточением и сопровождалась многочисленными убийствами и казнями. В борьбе истощились и погибли обе династии. Населению Англии война принесла усобицы, гнет налогов, расхищение казны, беззакония крупных феодалов, упадок торговли, прямые грабежи и реквизиции. В ходе войн значительная часть феодальной аристократии была истреблена, многочисленные конфискации земельных владений подорвали ее могущество. В то же время увеличились земельные владения и возросло влияние нового дворянства и торгового купеческого слоя, которые стали опорой абсолютизма Тюдоров.
1485 -- битва при Босфорте

\textbf{Литература:}
\begin{itemize}
	\item Jones W. G. York and Lancaster (1399-1485). London, 1914.

	\item Goodman A. The wars of the Roses: Military activity and English society, 1452-1497. London, 1981.

	\item Ross C. The wars of Roses: a concise history. London, 1986.

	\item The wars of the Roses: From Richard II to the fall of Richard III at Bosworth field seen through the eyes of their contemporaries. / Еd. by Hallam E. London, 1988.

	\item Pollard A. J. The wars of the Roses. London, 1988.
\end{itemize}
\subsection{Великие географические открытия}
XV - XVI --- Индийский океан и Атлантические океан 
Причины:
\begin{itemize}
	\item Перенаселение городов
	\item кризис торговли со странами Востока
	\item Нехватка золота
\end{itemize}

Открытия:
\begin{enumerate}
	\item Португалия. Со 2-ой четвертиXV веку начинает активное освоение Атлантики:
	\begin{itemize}
		\item острова Зелёного мыса
		\item Азорские острова
	\end{itemize}
	Все эти открытия связаны с именем Генриха Мореплавателя. 
	
	1488 -- Бартоломеу Диаш открывает дорогу до мыса Доброй Надежды
	
	1497 - 1499 -- экспедиция Васку да Гамма. Доплыл до Индии. Были людские и корабельные потери, но то, что он привёз окупило экспедиию в 20 раз.
	
	1500 -- Педру Кабрал случайно открывают Южную Америку (они были первыми)
	
	Основным направление португальцы избрали Индийский океан.
	
	колония -- далёкая территория, находящаяся под чьей-то властью
	
	фактория -- колония с целью добычи чего-либо
	\item Испания. В связи с шарообразностью земли, они поняли, что морской путь в Индию существует. 
	
	Христофор Колумб. Ему выделили 3 корабля.
	
	1492 -- начало экспедиции.
	
	Высаживаются на одном из островов Карибского архипелага
	
	Открытие территорий
	
	В следущие годы Колумб проведёт ещё 3 экспедиции. Сам колумб ещё не знает, что изучает Америку. Испанцы находят там золото
	
	1503 - 1504 -- Америго Веспуччи -- первый, кто сказал, что Америка это не Индия, а что-то другое. После этого по его запискам этот материк назвали Америкой

	1510 -- Больбоа. Высаживается на территории понамы, прошёл её насквозь и увидел Тихий Океан. Национальная валюта Понамы -- больбоа. Итог -- походы конкистодоров.
	
	1530-е -- Франсиско Писарро. Поход -- уничтожение Ингов. Золото у них было очень много, и оно не считалось чем-то ценным, а скорее украшением. Захватили вождя Ингов. Взяли с них выкуп -- заполненный золотом зал до высоты человека на лошади.
	
	Испания создаёт на территории Америки королевство новая Испания
	
	1504 -- между Испанией и Португалие был заключен договор, в котором они делили мир. вся Америка "досталась" Испании.
	
	1519 - 1522 -- Фернанн Могеллан. Первый человек, совершивший кругосветное путешествие. В ей он  погиб. Груза хватило, чтобы скомпенсировать всё.
пришло много золота.
\end{enumerate}

\subsection{Экономические преобразования}

Толчком стали великие географические открытия и кризис феодальной системы.

Важным следсвиями ВГО стали:
\begin{enumerate}
	\item приток золота, что привело к "революции цен". Эта революция цен привела с одной стороны к обогащению производителей. Но с другой привело к упадку феодалов и королевских служащих.
	\item межконтинентальная тороговля. 
	
	Плюсы : очень большая прибыль (5-25 раз)
	
	Минусы : высокая степень риска, пиратсво, корабль может утонуть..
	
	Их межконтинентальной торговли следует:
	\begin{enumerate}
		\item Упадок старых торговых центров (Венеция, Генуя, Константинопль)
		\item Формирование новых (морские города в Атлантическом океане: Лиссибон, Кадис, Сивилья, Лондон, Голандия (Нидерланды), Антверпен)
		\item Формирование бирж, рынка ценных бумаг, возникновение спекуляций (сделки на то, чего ещё нет; игры на повышение/понижение -- вы узнали, что прибыло лишь одна баржа с зерном. Вы можете скупить всё зерно, а затем продать их втридорога).
		\item Начинают формироватся торговые компании. Многие из гних пользовались покровительством государства. Такие комапнии зачастую получали право на монополию.
	\end{enumerate}
	\item Кризис цеховой системы. Определённый товар нужно делать только одинаковым образом. нельзя сделать отличный товар и продать его за отличные деньги, отсутсвие конкуренции.
	
	Люди понимают, что лучше организовывать малое производство, а затем лично продавать купленный товар.
	\item Мануфактурное производство. (как разделение труда) Нужны люди, которые умебт делать какой-то этап производство. Может быть централизованным (все трудятся под одной крышей) и рассеянным. 
	\item Формирование капитализма
	\item Формирование рыночных отношений в деревне или на селе
	\item Огораживание в Англии. Дворяне стали сгонять арендаторов и сдавать их род овечьи пастбища.


\subsection{Реформация в Европе}
Реформация -- процесс формирования новых течений в Христианстве.
Реформация -- движение за обновление и преобразование католичекой церкви в начале XVI века

Принципы:
\begin{enumerate}
	\item Падение авторитета церковной власти.
	\begin{itemize}
		\item Гуманистическвая философия
		\item Cтремление к благополучию
		\item Обвинение церковных иерархов (глав церкви) в их несоответствии христианскому служению. Цитата путешествинника "Если в мире существует ад, от он стоит прямл над Римом." 
		
		Александр VI (1492 -- 1503). Ходили слухи, что имел свою собчтвенную дочь в наложницах.
		
		Лев X (1513 -- 1521)
		
		И тот и другой папы
		\item Стремление контролировать монархов 
	\end{itemize}
	\item Начало торговли индульгенциями
	\item Высокие церковные налоги
	\item Внутренняя консервативность
\end{enumerate}

\end{enumerate}
\subsubsection{Начало}

Мартин Лютер 31.10.1517 -- "95 тезисов (против индульгенций)", город Виттенберг.

Лютеру повезло: священики были раздражающими (один раздавал индульгенции за будущие грехи).

Основы учения Лютераства:
\begin{enumerate}
	\item Пападание в рай или спасение души | ''Спасение верой'' -- Церковь, как посредник не нужна. Если человек по-настоящему верит, то онм будет добрым.
	\item Нет авторитета больше, чем свщенное писание -- в Лютеранстве любой \footnote{достаточно хорошо знающий Священное Писание} человек может вести проповедь (пастор)
	\item Отсутсвие монашества. Лютеране не стремились выделять себя из общей массы.
	\item Лютер выступал за дешёвую церковь. Без резных ангелов, позолоченные .. он приходит, чтобы общаться с богом. Редко ставят иконы. Редко ставят что-нибудь, кроме распятия. Также нет раскошных выходов, служб и т.п. Службы должны быть максимально скромными. Богослужение ведётся на национальном языке (в первую очередь немецком). Людям становится понятнее что вообще происходит.
	\item Отмена десятины
	\item Светская власть (монархи, князья, герцоги и т.д. главы государтсва) управляет церковью. Священники стали гос. служащими.
\end{enumerate}

1519 -- Лютер уничтожает папскую буллы (указ, документ, декрет). Его учение назвали ересью

1521 -- Вормс (Рейхстаг, суд). Отрекаться от учения он не собирался,и всё бы пошло плохо. если бы этим движением не заитересовались высшие слои общества. Лютер покидает суд и находит убежище в землях Саксонского герцогства. Лютеране vs КОтолики

Крестьянская война в Германии

Томас Мюнсер поднял крестьянскую войну. Лютер не поддерживал её.

1529 -- Карл V  в своём указе говорит о том, что отныне на территории Священной Римсой Империи запрещено исповедовать что-либо, кроме католичество

Народ протестует и от этого образуется термин Протестантизм.

1555 -- Итогшом реформации стал Аугсбургский религиозный мир. Карл V издаёт указ о том, что чья страна, того и религия. От монарха зависит является ли страна веротерпимой или радикальной. Тем не менее войны на религиозной почве будут происходить ещё 100 лет. 

Север -- Простестанский

Юг -- Католический

Норвегия, Швеция, Дания, Финяндия. Исландия приняли Лютеранство. Не нужно отчитываться перед папой, отдавать денег, религией управляет монарх

\subsubsection{Распространение}
Швейцария. В неё стали проникать протестанты из самых разных стран. Реформация в Швейцарии связана со священником Жаном Кальвином. Центром стал швейцарский город Женева. Начал проповедовать Кальвинизм

Кольвин говорил о том, что судьба человека предопределана ещё до его рождения. Если человек красивый и здоровый, значит он избранный богом. Если же нет, то его ждёт ад. Если ты вышел второго типа, то надо не опускать руки, а развиватся, и тогда станет понятно, что он избран богом к спасению.

Кольвинизм предполагал очень серьёзные бытовые ограничения: отменялись празднества, люди должны были постоянно трудится (развлечение -- молитвы). Честно заработанные деньги устремляли к раю. Из-за этого идеи Кольвина нашли признание у зарождаюшегося класса предпринимателей. В XVI веке распространились по Швейцарии, Голландии, Шотландии и Франции.

Англия. Генрих VIII (имел 6 жён). Первой женой была Екатерина Арагонская. Была только дочка. Нашёл другую жену. Хотел развестись с первой, но папа не дал добро.

1533 -- развод Генриху дал один из английских епископов

1533 - 1534 -- Англиканство: Король провозглашался главой церви в своём государстве, изъятие церковной земли и имущества (закрытие монастырей, около 3000), террор к инакомыслящим, в том числе Томас Мор, который отрицал развод генриха, его казнили, но он стал святым.

Весь XVI в Англии идут споры. Третьей женой была Джейн Сеймур.Эдуард VI (1537 -53) Престол перешёл к старшей дочери Марии (1553 - 58). Была принята попытка контрреформации. 

Мария Кровавая, Католичкеа, попыталась вернуть в стране католичество. Это также сопровождалось репрессиями.

Затем правила дочь от второго брака Елизавета I (1558 - 1603)

Франция 

дз

гугеноты -- приверженцы Кольвенизма.

Случай в Васси -- 1 марта 1562 года массовая резня французcких протестантов. Именно с этого события и влоть до принятия нансткого эдикта длится гугенотская война.

Варфоломеевсккая ночь -- 24 августа 1572 второе массовое убийство гугенотов. Была спровоцирована Екатериной Медичи.

Нантский эдикт -- закон, даровавший французским протестантам-гугенотам вероисповедные права. 

Париж стоит мессы — крылатое выражение, приписываемое Генриху Наваррскому в связи с его решением в 1593 году принять католичество, чтобы стать королём Франции под именем Генриха IV и основать династию французских Бурбонов. Выражение используется в качестве шутливого оправдания морального компромисса, сделанного в целях личной выгоды.

Генрих Бурбон -- лидер гугенотов в конце Религиозных войн во Франции, король Наварры с 1572 года (как Генрих III), король Франции с 1589 года (формально — с 1594), основатель французской королевской династии Бурбонов.

Права Генриха IV на трон были подтверждены Генрихом III, который, будучи смертельно ранен, приказал своим сторонникам присягать наваррскому монарху, однако стать королём Франции он смог только после длительной борьбы. Для того чтобы нейтрализовать своих соперников, 25 июля 1593 года Генрих Наваррский принял католицизм и уже 22 марта 1594 года вступил в Париж. В 1595 году папа римский даровал Генриху отпущение, сняв с него отлучение от церкви и провозглашение еретиком. Для прекращения межконфессиональной вражды Генрих IV 13 апреля 1598 года подписал Нантский эдикт, даровавший свободу вероисповедания протестантам, вскоре после этого Гугенотские войны закончились.

Екатерина Медичи

etc. Карл X Генрих III, маргарита Валуа, Франсуа Гиз

\begin{Huge}
\textbf{ДЗ -- зачёт за четверть (
\begin{enumerate}
	\item Иван III, Василий III
	\item Столетняя война
	\item Великие географичекие открытия
	\item *Реформация
\end{enumerate}}


\end{Huge}

)

\section{Европейские революции XVI -- XVII}
\begin{enumerate}
	\item Нидерландская революция (война за независимость)	
	\begin{itemize}
		\item Крупный экономический центр
		\item 4 млн человек, 300 городов (Амстердам, Роттердам, Антверпен, Гаага, Брюссель, Брюгге)
		\item к середине XVI века распространяется протестантизм (кальвинистуого толка)
		\item Нидерланды -- испанское владение
		\item в это время 17 провинций
	\end{itemize}
	Причины:
	\begin{enumerate}
		\item Усиление контроля (штатгальтер -- наместник)
		\item Испанский запрет торговли с Англией (что приносило огромные убытки Голландским торговцам.)
		\item Увеличение налогов в Нидерландах
	\end{enumerate}
	После того , как на престол взошёл Филлипп II в 1556 году в Нидерланды вводтся испанские войска 
	
	1560-е годы -- антииспанское сопротивление. Лидерами этого движения становятся Вильгельм I Оранский и Граф Эгмон.
	
	Основными требованиями были:
	\begin{enumerate}
		\item вывод испанских войск
		\item созыв местного Нидерландского парламента (Федеральных штатов)
		\item Самостоятельность Нидерланд
	\end{enumerate}
	С 1566 начинаются восстания.
	
	Испанский король выделяет дополнительное войско. Филлип II назначает главой французских войск Герцога Альбу. (его действия в Голландии сопровождаются террором)
	
	Кровавый совет. Он приговорил к смерти Эгмона.
	
	Формируются отряды повстанцев -- Гёзы (нищие)
	
	1572 -- Взятие Брилле гёзами
	
	Лидерам повстанцев наначают Вильгелма Оранского.
	
	К середине 1570-х годов Испания сталкивается с финансовыми трудностями.
	
	1576 -- "Испанская ярость" в Антверпене
	
	Герцог Альба перестал быть главнокомандующим из-за большого количества неудач
	
	в конце 1570-х подписывается две Унии:
	\begin{itemize}
		\item Утрехтская Уния -- 7 провинций (север) $\rightarrow$ Республика Северных Провинций $\rightarrow$ Голлаендская республика.
		\item Аррасская -- 10 провинций (юг) $\rightarrow$ Современная Бельгия
	\end{itemize}
	
	Войны на территории Нидерландов продолжаются до 1609, перемирие 12 лет
	
	Окончательно Испания признаёт независимость Нидерландов лишь в 1648.
	
	Итоги :
	\begin{itemize}
		\item Независимость Северных Нидерландов
		\item крупнейший экономический центр ("Тюльпаномания")
		\item Огромный флот $\rightarrow$ Создание колоний.
		\item научный центр (Левенгук, Тульп)
	\end{itemize}	 
	\item Английская революция 1642 -- 1660
	
	1603 -- 1625 Яков I
	
	Важные моменты прадения:
	\begin{itemize}
		\item с 24 июля 1567 года начал править страной под опекой регентского совета
		\item с 12 марта 1578 года -- единолично
		\item Яков хотел единолично управлять страной, т.к. парламент часто отклонял его предложения.
		\item в 1607-8 совместно с Шотландией завоевали Ирландию
		\item в 1604 подписал Лондонский мир с Испанией, звершив англо-испанскую войну.
		\item Укрепившиеся хорошие отношения между Испанией и Англией почти обеспечили брак между Карлом и испанской инфанты.
		\item 27 марта 1625 -- конец правления
	\end{itemize}
	1625 -- 1649 Карл I
	
	Династия Стюартов
	Причины:
	\begin{itemize}
		\item Ухудшение положения бедных слоёв (огораживание, снижение цен на драгоценные металлы)
		\item Конфликт королей с парламетом (стремление к единоличному правлению)
		\item Экономическая политика королей (стремление получить деньги любым способом, например, созданием новых титулов; опора на монополии, опора на цехи)
		\item Коррупция	
	\end{itemize}
	
	1628 -- принимается "Петиция о праве"
	
	Укреплялись позиции пуритан (движение за очищение англиканской церкви)
	
	1640 -- Карл I созывает парламент с целью введения новых налогов. "Долгий парламент" (распустили на 13 лет). Члены парламента выдвинули следующие требования: снижение влияние на суды, нет созданию королевской армии, коонтроль парламента над церквью
	
	1640 - 42 -- споры в парламенте
	
	1642 -- Карл I покидает Лондон и отправляется в Нотингем и созывает своих торонников для борьбы с парламентом
	
	Формируется два лагеря:
	\begin{itemize}
		\item Роялисты (ковалеры (железнобокие)) 
		\item Парламентарии (круглоголовые)
	\end{itemize}
	
	Война шла с переменным успехом, но после победы при Нейсби в 1645, фактически это означчает победу парламента
	
	Во многом своей победой парламет обязан "армии нового образца", в которой устранялись сословные преграды, было разделение на рода войск, формирование на добровольных началах
	
	1646 -- Карл I попадает в плен.
	
	в конце 1647 -- вопрос о суде короля
	
	очень долгое время парламент не решает судить короля, а когда дело таки до этого дошло, то оказалось недостаточно голосов для его казни
	
	1648 -- полковник Прайд проводит Прайдовую чистку -- изшнание из парламента умеренных депутатов, выступавших против казни короля
	
	январь 1649 -- финальная стадия суда
	
	30 января 1649 -- казнь Карла I, через декапитацию (отруьления головы). Первый случай публичной казни короля.
	
	1649 -- Англия провозглашена республикой
	
	Создаётся советный парламент, куда входят военные. Реальным руководителем этого советом становится Оливер Кронвель
	
	Начинается период правления Кронвеля. Он расправляется с опоззиционными ему силами:
	\begin{itemize}
		\item Левеллерами (выступавшими за всеобщее равенство)
		\item Диггерами (занимавшие пустующие земли)
	\end{itemize}
	
	1651 -- навигационный акт (торговать зарубежными товарами позволялось лишь Английскийскому суда (корабли) и суда страны производителя) 
	
	Войны с Голландией. Псле них, Англия укрепилась, как первая морская дрежава
	
	Долгий парламент был распущен, а Кронвель был провозглашён лордом протектором (фактически с королевскими полномочиями)
	
	Кронвель также ведёт войны против Ирландии и Шотландии
	
	1658 -- смерть Кронвеля. Приемником становится его сын. Он ен правит так жёстко и вновь созывается парламент. По его решению вновь призван король
	
	На престолы приглашён Карл II Стюарт. Ресавация Стюартов
	
\end{enumerate} 

\section{30-и летняя война}

1-ая общеевропейская война

\begin{tabular}{|l||l|}
\hline
Католики(сторонники императора) & Протестанты (противники императора)\\
\hline 
"Католическая лига" & "Евангелическая уния"\\
(союз немецких князей католиков) & (союз немецких князей-протестантов)\\
Священная Римская Империя & Чехия\\
Испания & Дания\\
Польша & Щвеция\\
 & \underline{Франция}(она католическая)\\
 & Англия\\
 & Нидерланды\\
 & Россия\\
 & \\
\hline
\end{tabular}

1618 -- Пражская дефенестрация 

1618 -- 1625 -- Чешский период войны

1625 -- 1629 -- Датский период войны

1630 --  Густоф II Адольф. Был очень смелым . Погиб, попав в дыму под обстрел.



Шведская армия немногочислена, но хорошо оснащена, выучена и дисциплинирована

1631 -- битва шведов при Брейто

1632 -- Лютцен

1632-34 -- Смоленская война, Россия выступает на стороне противников императора. Россия пытается вернуть Польше город Смоленск, потерянный в ходе смутного времени. закончилась неудачей. "козлом отпущения" стал Михаил Шейх. Также отвлеки силы Польши, чтобы она не помогала императору.

Россия пославляла Швеции продовольствие в ходе войны

Воначальник С.Р.И. Альбрехт Валенштайн. Его войско кормило само мебя (грабёж и мародёрство)

Франция -- катлическое государство, но тем не менее встаёт на сторону противников императора.

Самым главным идеологом стал каринал Арман Решелье (ё). Франция должна искать выгоду в войне. На первых порах всё было скаладно

1634 -- первоё серьёзное поражение. Нордлинген

1635 -- по настоянию кардинала Решилье в войну вступает в Францию

в 1640 активность боевых дейтсвий ослабла, так как страны были истощены, и стали понемногу искать благоприятный конец.

Борьба Каталонии за независимость.

1643 -- битва при Рокруа. Французы против Испанцев. Командовал французсккими войсками такой человек, как Конде. Он грамотно оценив ситуацию и использовав опыт войны прошлых лет. Он понял, что лучше строить мощную пехоты в терции. Испанцы вынуждены отступать

1648 -- Французкая армия под командованием Кнде наносит последнее поражение в битве при Лансе.

Подписывается Вестфальский мир (договор) 

Шведы получили рад земельн асевере Германии (померания). были заблокированы важные реки Германии. Швеция на 7 лет становиться всеевропейской империей. Поллоный контроль над Балтийским морем

Французы получают радя приграничных территорий. Также становиться лидирующей европейской державой.

Независимость Нидерландов и Швейцарии

Больше всего пострадала С.Р.И. Князья получали самостоятельность во внешней политике. Фактически получают независимость. Оговаривалось, что никакие союзы не должны заклюяаться против императора. 

до 1806 С.Р.И. как государства уже фактически не существовало.

Последущие 150 лет С.Р.И. существует лишь формально

правитель мог определять какая у него на территории религия. А значит теперь на территории С.Р.И. теперь свобода религии. В некоторых случаях, как например в Вене, было понятно, что там может быть только католицизм. Но в других "государствах" мог быть и протестантизм

Это была последняя война с религиозным подтекстом  в Европе


Потери оцениваются от 5 до 7 погибших

В некоторых провинциях Германии потери -- 70\% населения

\section{Франция в XVII веке, Людовик XIV}

1638 -- 1643 -- 1715 -- Людовик XIV -- сын Людовика XIII и А.Австрийской

1634 -- вступление не престол. Правит регенский совет вмсте с первым министром Франции карлиналом Мазарини (итальянец).

Мазарини фигура крайне не популярная (не чужд собирать дополнительные налоги и незаконно обогащаться, ходили слухи и любовной связи с матерью Людовика -- Анной).

1648 -- 52 -- мятеж Фронда (выступать против, фрондировать) (Конде и Тюренн). Они возмущены отношением регенсткого совета к ним и говорят о том, что не хотят подчиняться Мазарини.

4 года и идёт гражданская война. Есть разорение и опустошение фарнцузским земель.

лидеров Фронды вели личные интеречы по изгнанию Мазарини. некоторые называли Фронду пародией на английскую революцию

Фрондировать -- высказывать недовольство, но решительных действий не предпринимать

Конде официальо простили в 59, а приняли обратно на двор только в 71

события Фронды, а также правление Мазарини, урепили Людовика в необходимости единовластия

1666 -- Мазарини умер

был создан государственный совет. Многие были уверены, что Людовик созовёт певого министра, но этого не случилось.

Фраза Людовика "Государство -- это я!"

во Франции утверждается абсолютизм

Жан Батист Кольер -- один из главных советников короля, инициатор финансовой, налоговой, административной и морской (введение поморской повинности) реформ. Приведение гос. аппарата в нужную форму, отмена старых и изобретение новых налогов. 

Маркиз де Ливуа -- инициатор военной реформы. Франция делала лучшую в Европе артиллерию. удалось нарастить армию. Офицеров пришлось заново дисциплинировать. Продвигали не родовитых, но прославившихся подвигами людей. 80 -- 100 людей готовых выступить в любой момент.

титул Людовика -- "Король -- Солнце". Считается, что во время одного из балов он выступал в своей пьесе, как человек в золотистых доспехах. С другой сторны, он центр Франции. Или то, что он вокруг себя создал "модель мира" из советников

\subsection{Внешняя политика}
\begin{enumerate}
\item Уже слабеющая Испания всё ещё проблема. У неё есть владения в Нидерландах.

Найти повод для войны не было проблемой

1667 -- 68 -- война с Испанией за города в Нидерландах (сейчас Бельгия)
\item Голландия. Франция видит в ней соперникаи объявляет ей войны, но за Нидерланды вступает ряд других стран, включая Империю(Австрия)

Франции всё равно удаётся одержать победу\footnote{1673 -- штурм Голлндской крепости Маастрихт. Людовик, лично находившийся на месте, был командиром. Командиров королевских мушкетёров был д'Артаньян. Он погиб при штурме. Людовик знал его, как человека, который служил ещё его отцу.}

Война закончилась в 1678 году. Нимвегенский

По итогам войны Франции отошла Латарингия (кусок Империи)

1673 -- создание "Комиссии по воссоединению"

комиммия проработала 5 лет. Страны стали высказывать своё негодование. Страссбург и Эльзас были просто объявлены владениями Франции. 

с 95 по 98 Франция вновь воюет с Империей, но уже вынуждена ограничением статус-кво и не выигрывает.

\end{enumerate}

\subsection{Внутрення политика и Lвор Людовика XIV}

Начинает вводить большое количество налогов. Это сказывается на положении низших слоёв, в первую очередь крестьян, начинает ухудшаться. Начинает закладываться база для революции.

В своей религиозной политике Людовик придерживался католической традиции. Воспитывался изуидами, поэтому был довольно фанатичным. Но поскольку позиция гугенотов(французских протестантов) не сильно ослабилась. В один момент они надоели ему и в 169 был подписан эдикт Фонтенбло, оменявший действие Нантского эдикта о религиозной терпимости.

Итогом этого эдикта стало бегсиво приемрно 200 000 гугенотов. Они были крупными, часто представителями буржуазии. Бежели они в Северную Америку и основали ВБ

Именно при Людовике начинает расти значимость его двора.

В конце 60-х годов двор Людовика переехжает в Версальский дворец.

Самый большой королевский парк вообще.

Существовали сложные правила поведения, зависящие от положения при дворе.

особое влияние на Людовика оказывали фаворитки.

К концу правления семья Людовика была довольно большой

После 1711 год династию Бурбонов настиг кризис. До 1714 скончался и сын и двое внуков и некоторое количсестыо правнуков. В итоге на момент смерти Людовика остался только правнук. 

\subsection{Война за Испанское наследство}

1700 -- смерть короля Испании. За наследие короны начина.т бороться те, кто имееют хоть какую-то связь. Спорили: Фарнция и Австри (Империя).

Было принято, что корону унаследует внук Людовика XIV -- Филипп Анжуйский

Против этого выступили правтически все Европейские страны

Валлар (Фр.) vs Евгений Савойский (Империя) + герцог Мальборо

1709 -- битва при мальплаке. Самое крупное по численности сражение в XVIII веке

Наибольшую выгоду получает Англия, которая начинает становиться главной Европейской державой

Людовик XV -- 3-х летний. Ему завещается огромное количество нерешённым проблем.

будет проверочная по Людовику и 3-и летней войне

\section{Иван IV грозный}

1530-1533-1584

\subsection{Детсво и юность}
Василий III + Елена Глинская

Есть легенда, что в день его рождения ударила молния в кремль и он сгорел

как раз при нём Русское государство окончательно централизовалось (Псков, Рязань и Смоленск)

воообще рождается два ребёнка, но его брат считался умственно отсталым

1533 -- Василий III умирает и Иван IV становится правителем. Василий понимает, что оставляет семью в тяжёлом положении

Назначается опекунский совет. Фактически его возглавила Елена Глинская.

1534 -- Елена приговаривает к смерти Михаила Глинского (своего дядю)

1534 -- 1536 -- о своём нипоаиновении Ивану заявляют братья Василия III (Юрий и Андрей Старицкий)

Елена подавляет их выступление. Почти сразу в тюрьму помещают Юрия, где он и помирает.

1537 -- Андрей Старицкий тоже попадает в тюрьму и он тоже там умирает

Елена имела фаворита -- Овчино-Теленев-Оболенский


1538 -- смерть Елены. Позже выяснют большое количество тяжёлых металлов (она возможно была отравлена). Начинается фактическое боярское правление(Шуйские, ...)

1543 -- первое выступление Василия IV против бояр. Он приказывает Шуйского отправить на псарню (на обгладание). Все остальные бояре юыли туда приглашены. Казнь Шуйского.

Иван уже собирал вокруг себя людей, которые, как ему казалось, были к нему максимально близки

Уже в юношеском возрасте наиболее влиятельным чловеком становится Митрополит Макарий. Занимался образованием князя и со временем стал очень близок. Начинает подсказывать илеи по правлению. Рассказывает идею царства. Пока что он князь.

Иван IV -- потомок Византийской имерии (Софья Полеонтолог)

Параллельно с этим развивается идея Филофея "Москва -- III Рим"

И тот и другой продвигают идея, что хранитель христианства -- Московско-Русское государство 

1547 -- венчание на царство. Его теперь звали Иваном I. Успенский собор Московского кремля. Здесь же будут венчаться все остальные

почти сраз признаётся императором Римской Империи 

1547 -- первая женитьба. На Анастасии Захарина-Юрьевой. Он её очень любит. 

в браке будет 3-е детей:
$\begin{cases}$	
	Дмитрий 1552-53$\\
	$Иван 1554-81 $\\
	$Фёдор 1556-1598$\\
\end{cases}$

\subsection{Рефоры избранной рады}

Сильвестр, Ал. Адашев, князь Андрей Курбский

Созыв первого Зеского собора -- сословно-представительский орган.

Провозглашается необходимость реформ

1550 -- издание "Нового судебника". ограничение Юрьева дня

Размеры подилого тоже увеличились

именение налогообложения.

Саха -- территория, объединение большого количесво крестьянских дворов. одновременно со всех собирались налогов.

Изменялось положение сомоуправления. Чиновники могут сами решать некоторые мелкие дела. 


\begin{Huge}
\begin{center}выбираем книгу по эпохе Ивана Грозного (не литературные). читать её. 

во введении писать сжатую харакетристику Ивана Грозного (чему посвщена книга) оочень кратко .. на лист А4

Основная часть -- такаято глава посвящена тому то в ней автор расписывает такие то такие то такие то события (краткий пересказ)+ автор отдельна пасписывает то то то то, в частности то то то то $+^*$ недоговки (если например автор во время рассказа "забыл" что-то написать)

Заключение -- отхыв от себя. Интересность/степень подачи


7-8 страниц (A4 со вступление. 14 Times 1.5--отступ)

дедлайн -- 13 марта 0:01

aapotyomkin@gmail.com


\end{center}
\end{Huge}
.

.

.

.

.

.

\end{document}